\begin{code}[hide]%
\>[0]\AgdaKeyword{module}%
\>[0I]\AgdaModule{MS}\<%
\\
\>[0I][@{}l@{\AgdaIndent{0}}]%
\>[10]\AgdaSymbol{(}\AgdaBound{e}\AgdaSpace{}%
\AgdaSymbol{:}\AgdaSpace{}%
\AgdaPrimitive{Set}\AgdaSymbol{)}\AgdaSpace{}%
\AgdaKeyword{where}\<%
\\
\>[0]\AgdaKeyword{open}\AgdaSpace{}%
\AgdaKeyword{import}\AgdaSpace{}%
\AgdaModule{Data.Product}\AgdaSpace{}%
\AgdaKeyword{using}\AgdaSpace{}%
\AgdaSymbol{(}\AgdaRecord{Σ}\AgdaSymbol{;}\AgdaSpace{}%
\AgdaOperator{\AgdaFunction{\AgdaUnderscore{}×\AgdaUnderscore{}}}\AgdaSymbol{;}\AgdaSpace{}%
\AgdaOperator{\AgdaInductiveConstructor{\AgdaUnderscore{},\AgdaUnderscore{}}}\AgdaSymbol{;}\AgdaSpace{}%
\AgdaField{proj₁}\AgdaSymbol{;}\AgdaSpace{}%
\AgdaField{proj₂}\AgdaSymbol{;}\AgdaSpace{}%
\AgdaFunction{∃}\AgdaSymbol{;}\AgdaSpace{}%
\AgdaFunction{Σ-syntax}\AgdaSymbol{;}\AgdaSpace{}%
\AgdaFunction{∃-syntax}\AgdaSymbol{)}\<%
\\
\>[0]\AgdaComment{-- example from}\<%
\\
\>[0]\AgdaComment{-- https://pdfs.semanticscholar.org/f94b/268c1c91dd1de22cf978a7ea03f8860cbe9d.pdf}\<%
\end{code}

We assign Agda types to the categories, knowing that entities are postulated as a
type.

\begin{code}%
\>[0]\AgdaFunction{t}\AgdaSpace{}%
\AgdaSymbol{=}\AgdaSpace{}%
\AgdaPrimitive{Set}\<%
\\
\>[0]\AgdaFunction{S}\AgdaSpace{}%
\AgdaSymbol{=}\AgdaSpace{}%
\AgdaFunction{t}\<%
\\
\>[0]\AgdaFunction{NP}\AgdaSpace{}%
\AgdaSymbol{=}\AgdaSpace{}%
\AgdaSymbol{(}\AgdaBound{e}\AgdaSpace{}%
\AgdaSymbol{→}\AgdaSpace{}%
\AgdaFunction{t}\AgdaSymbol{)}\AgdaSpace{}%
\AgdaSymbol{→}\AgdaSpace{}%
\AgdaFunction{t}\<%
\\
\>[0]\AgdaFunction{VP}\AgdaSpace{}%
\AgdaSymbol{=}\AgdaSpace{}%
\AgdaFunction{NP}\AgdaSpace{}%
\AgdaSymbol{→}\AgdaSpace{}%
\AgdaFunction{t}\<%
\\
\>[0]\AgdaFunction{V}\AgdaSpace{}%
\AgdaSymbol{=}\AgdaSpace{}%
\AgdaFunction{NP}\AgdaSpace{}%
\AgdaSymbol{→}\AgdaSpace{}%
\AgdaFunction{NP}\AgdaSpace{}%
\AgdaSymbol{→}\AgdaSpace{}%
\AgdaFunction{t}\<%
\\
\>[0]\AgdaFunction{Det}\AgdaSpace{}%
\AgdaSymbol{=}\AgdaSpace{}%
\AgdaSymbol{(}\AgdaBound{e}\AgdaSpace{}%
\AgdaSymbol{→}\AgdaSpace{}%
\AgdaFunction{t}\AgdaSymbol{)}\AgdaSpace{}%
\AgdaSymbol{→}\AgdaSpace{}%
\AgdaSymbol{(}\AgdaBound{e}\AgdaSpace{}%
\AgdaSymbol{→}\AgdaSpace{}%
\AgdaFunction{t}\AgdaSymbol{)}\AgdaSpace{}%
\AgdaSymbol{→}\AgdaSpace{}%
\AgdaFunction{t}\<%
\\
\>[0]\AgdaFunction{N}\AgdaSpace{}%
\AgdaSymbol{=}\AgdaSpace{}%
\AgdaBound{e}\AgdaSpace{}%
\AgdaSymbol{→}\AgdaSpace{}%
\AgdaFunction{t}\<%
\end{code}

Agda's equality symbol indicates our the semantic interpretation,
$\llbracket\_\rrbracket$, of our grammar's categories, whereby we are here
working with a shallow embedding. One could instead elect to define both the GF
embedding as a record and Montague's intentional type theory as an inductive
dataype, whereby the semantics could then be given explicitly. This degree of
formality is not necessary for our simple examples, but doing so would allow one
to prove meta-properties about the GF embedding, and should certainly be
investigated. The two GF functions for forming sentences and verb phrases can
then be given the following Agda interpretations.

\begin{code}%
\>[0]\AgdaFunction{sentence}\AgdaSpace{}%
\AgdaSymbol{:}\AgdaSpace{}%
\AgdaFunction{NP}\AgdaSpace{}%
\AgdaSymbol{→}\AgdaSpace{}%
\AgdaFunction{VP}\AgdaSpace{}%
\AgdaSymbol{→}\AgdaSpace{}%
\AgdaFunction{S}\<%
\\
\>[0]\AgdaFunction{sentence}\AgdaSpace{}%
\AgdaBound{S}\AgdaSpace{}%
\AgdaBound{V}\AgdaSpace{}%
\AgdaSymbol{=}\AgdaSpace{}%
\AgdaBound{V}\AgdaSpace{}%
\AgdaBound{S}\<%
\\
%
\\[\AgdaEmptyExtraSkip]%
\>[0]\AgdaFunction{verbp}\AgdaSpace{}%
\AgdaSymbol{:}\AgdaSpace{}%
\AgdaFunction{V}\AgdaSpace{}%
\AgdaSymbol{→}\AgdaSpace{}%
\AgdaFunction{NP}\AgdaSpace{}%
\AgdaSymbol{→}\AgdaSpace{}%
\AgdaFunction{VP}\<%
\\
\>[0]\AgdaFunction{verbp}\AgdaSpace{}%
\AgdaBound{V}\AgdaSpace{}%
\AgdaBound{O}\AgdaSpace{}%
\AgdaBound{S}\AgdaSpace{}%
\AgdaSymbol{=}\AgdaSpace{}%
\AgdaBound{V}\AgdaSpace{}%
\AgdaBound{S}\AgdaSpace{}%
\AgdaBound{O}\<%
\end{code}

\begin{code}[hide]%
\>[0]\AgdaFunction{nounp}\AgdaSpace{}%
\AgdaSymbol{:}\AgdaSpace{}%
\AgdaFunction{Det}\AgdaSpace{}%
\AgdaSymbol{→}\AgdaSpace{}%
\AgdaFunction{N}\AgdaSpace{}%
\AgdaSymbol{→}\AgdaSpace{}%
\AgdaFunction{NP}\<%
\\
\>[0]\AgdaFunction{nounp}\AgdaSpace{}%
\AgdaBound{D}\AgdaSpace{}%
\AgdaBound{N}\AgdaSpace{}%
\AgdaSymbol{=}\AgdaSpace{}%
\AgdaBound{D}\AgdaSpace{}%
\AgdaBound{N}\<%
\end{code}

We axiomatically include primitive lexical items of \term{love}, an entity
\term{j} for John, and \term{man} and \term{woman} as nouns, so that we can
articulate logically interesting facts about them. As our encoding of noun
phrases takes an argument, we can define the syntactic notion \term{John} by
applying an argument function to \term{j}.

\begin{code}%
\>[0]\AgdaKeyword{postulate}\<%
\\
\>[0][@{}l@{\AgdaIndent{0}}]%
\>[2]\AgdaPostulate{love}\AgdaSpace{}%
\AgdaSymbol{:}\AgdaSpace{}%
\AgdaBound{e}\AgdaSpace{}%
\AgdaSymbol{→}\AgdaSpace{}%
\AgdaBound{e}\AgdaSpace{}%
\AgdaSymbol{→}\AgdaSpace{}%
\AgdaFunction{t}\<%
\\
%
\>[2]\AgdaPostulate{j}\AgdaSpace{}%
\AgdaSymbol{:}\AgdaSpace{}%
\AgdaBound{e}\<%
\\
%
\>[2]\AgdaPostulate{man}\AgdaSpace{}%
\AgdaSymbol{:}\AgdaSpace{}%
\AgdaFunction{N}\<%
\\
%
\>[2]\AgdaPostulate{woman}\AgdaSpace{}%
\AgdaSymbol{:}\AgdaSpace{}%
\AgdaFunction{N}\<%
\\
%
\>[2]\AgdaPostulate{johnMan}\AgdaSpace{}%
\AgdaSymbol{:}\AgdaSpace{}%
\AgdaPostulate{man}\AgdaSpace{}%
\AgdaPostulate{j}\<%
\\
%
\\[\AgdaEmptyExtraSkip]%
\>[0]\AgdaFunction{John}\AgdaSpace{}%
\AgdaSymbol{:}\AgdaSpace{}%
\AgdaFunction{NP}\<%
\\
\>[0]\AgdaFunction{John}\AgdaSpace{}%
\AgdaBound{P}\AgdaSpace{}%
\AgdaSymbol{=}\AgdaSpace{}%
\AgdaBound{P}\AgdaSpace{}%
\AgdaPostulate{j}\<%
\end{code}

We define our quantifiers using the Agda's dependent $\Pi$ and $\Sigma$ which
make up the core of any dependently typed language.

\begin{code}%
\>[0]\AgdaFunction{Every}\AgdaSpace{}%
\AgdaSymbol{:}\AgdaSpace{}%
\AgdaFunction{Det}\<%
\\
\>[0]\AgdaFunction{Every}\AgdaSpace{}%
\AgdaBound{P}\AgdaSpace{}%
\AgdaBound{Q}\AgdaSpace{}%
\AgdaSymbol{=}\AgdaSpace{}%
\AgdaSymbol{(}\AgdaBound{x}\AgdaSpace{}%
\AgdaSymbol{:}\AgdaSpace{}%
\AgdaBound{e}\AgdaSymbol{)}\AgdaSpace{}%
\AgdaSymbol{→}\AgdaSpace{}%
\AgdaBound{P}\AgdaSpace{}%
\AgdaBound{x}\AgdaSpace{}%
\AgdaSymbol{→}\AgdaSpace{}%
\AgdaBound{Q}\AgdaSpace{}%
\AgdaBound{x}\<%
\\
%
\\[\AgdaEmptyExtraSkip]%
\>[0]\AgdaFunction{A}\AgdaSpace{}%
\AgdaSymbol{:}\AgdaSpace{}%
\AgdaFunction{Det}\<%
\\
\>[0]\AgdaFunction{A}\AgdaSpace{}%
\AgdaBound{cn}\AgdaSpace{}%
\AgdaBound{vp}\AgdaSpace{}%
\AgdaSymbol{=}\AgdaSpace{}%
\AgdaFunction{Σ[}\AgdaSpace{}%
\AgdaBound{x}\AgdaSpace{}%
\AgdaFunction{∈}\AgdaSpace{}%
\AgdaBound{e}\AgdaSpace{}%
\AgdaFunction{]}\AgdaSpace{}%
\AgdaSymbol{(}\AgdaBound{cn}\AgdaSpace{}%
\AgdaBound{x}\AgdaSpace{}%
\AgdaOperator{\AgdaFunction{×}}\AgdaSpace{}%
\AgdaBound{vp}\AgdaSpace{}%
\AgdaBound{x}\AgdaSymbol{)}\<%
\end{code}

We also notice that we may define the ditransitive verb \term{loves} two ways,
which are obtained commuting the NP arguments.

\begin{code}%
\>[0]\AgdaFunction{loves}\AgdaSpace{}%
\AgdaFunction{loves'}\AgdaSpace{}%
\AgdaSymbol{:}\AgdaSpace{}%
\AgdaFunction{V}\<%
\\
\>[0]\AgdaFunction{loves}\AgdaSpace{}%
\AgdaBound{O}\AgdaSpace{}%
\AgdaBound{S}\AgdaSpace{}%
\AgdaSymbol{=}\AgdaSpace{}%
\AgdaBound{O}\AgdaSpace{}%
\AgdaSymbol{(λ}\AgdaSpace{}%
\AgdaBound{x}\AgdaSpace{}%
\AgdaSymbol{→}\AgdaSpace{}%
\AgdaBound{S}\AgdaSpace{}%
\AgdaSymbol{λ}\AgdaSpace{}%
\AgdaBound{y}\AgdaSpace{}%
\AgdaSymbol{→}\AgdaSpace{}%
\AgdaPostulate{love}\AgdaSpace{}%
\AgdaBound{x}\AgdaSpace{}%
\AgdaBound{y}\AgdaSymbol{)}%
\>[42]\AgdaComment{-- (i)}\<%
\\
\>[0]\AgdaFunction{loves'}\AgdaSpace{}%
\AgdaBound{O}\AgdaSpace{}%
\AgdaBound{S}\AgdaSpace{}%
\AgdaSymbol{=}\AgdaSpace{}%
\AgdaBound{S}\AgdaSpace{}%
\AgdaSymbol{(λ}\AgdaSpace{}%
\AgdaBound{x}\AgdaSpace{}%
\AgdaSymbol{→}\AgdaSpace{}%
\AgdaBound{O}\AgdaSpace{}%
\AgdaSymbol{λ}\AgdaSpace{}%
\AgdaBound{y}\AgdaSpace{}%
\AgdaSymbol{→}\AgdaSpace{}%
\AgdaPostulate{love}\AgdaSpace{}%
\AgdaBound{x}\AgdaSpace{}%
\AgdaBound{y}\AgdaSymbol{)}%
\>[42]\AgdaComment{-- (ii)}\<%
\end{code}

We can then observe two different semantic interpretations of the phrase ``every
man loves a woman".

\begin{code}%
\>[0]\AgdaFunction{everyManLovesAWoman}\AgdaSpace{}%
\AgdaSymbol{=}\AgdaSpace{}%
\AgdaFunction{sentence}\AgdaSpace{}%
\AgdaSymbol{(}\AgdaFunction{Every}\AgdaSpace{}%
\AgdaPostulate{man}\AgdaSymbol{)}\AgdaSpace{}%
\AgdaSymbol{(}\AgdaFunction{verbp}\AgdaSpace{}%
\AgdaFunction{loves}\AgdaSpace{}%
\AgdaSymbol{(}\AgdaFunction{A}\AgdaSpace{}%
\AgdaPostulate{woman}\AgdaSymbol{))}%
\>[69]\AgdaComment{-- (i)}\<%
\\
\>[0]\AgdaFunction{everyManLovesAWoman'}\AgdaSpace{}%
\AgdaSymbol{=}\AgdaSpace{}%
\AgdaFunction{sentence}\AgdaSpace{}%
\AgdaSymbol{(}\AgdaFunction{Every}\AgdaSpace{}%
\AgdaPostulate{man}\AgdaSymbol{)}\AgdaSpace{}%
\AgdaSymbol{(}\AgdaFunction{verbp}\AgdaSpace{}%
\AgdaFunction{loves'}\AgdaSpace{}%
\AgdaSymbol{(}\AgdaFunction{A}\AgdaSpace{}%
\AgdaPostulate{woman}\AgdaSymbol{))}\AgdaSpace{}%
\AgdaComment{-- (ii)}\<%
\end{code}

The two interpretations of ``love" which differ as to where the respective
arguments are applied in the program, manifest in a semantic ambiguity of
whether there is one or possibly many women are in the context of consideration.
With Agda's help in normalizing these two types, this ambiguity is reflected in
whether the outermost quantifier is universal or existential. Note the product
type has been desugared to a non-dependent Σ-type.

\begin{code}%
\>[0]\AgdaSymbol{(}\AgdaFunction{i}\AgdaSymbol{)}%
\>[5]\AgdaSymbol{=}\AgdaSpace{}%
\AgdaSymbol{(}\AgdaBound{x}\AgdaSpace{}%
\AgdaSymbol{:}\AgdaSpace{}%
\AgdaBound{e}\AgdaSymbol{)}\AgdaSpace{}%
\AgdaSymbol{→}\AgdaSpace{}%
\AgdaPostulate{man}\AgdaSpace{}%
\AgdaBound{x}\AgdaSpace{}%
\AgdaSymbol{→}\AgdaSpace{}%
\AgdaRecord{Σ}\AgdaSpace{}%
\AgdaBound{e}\AgdaSpace{}%
\AgdaSymbol{(λ}\AgdaSpace{}%
\AgdaBound{x₁}\AgdaSpace{}%
\AgdaSymbol{→}\AgdaSpace{}%
\AgdaRecord{Σ}\AgdaSpace{}%
\AgdaSymbol{(}\AgdaPostulate{woman}\AgdaSpace{}%
\AgdaBound{x₁}\AgdaSymbol{)}\AgdaSpace{}%
\AgdaSymbol{(λ}\AgdaSpace{}%
\AgdaBound{x₂}\AgdaSpace{}%
\AgdaSymbol{→}\AgdaSpace{}%
\AgdaPostulate{love}\AgdaSpace{}%
\AgdaBound{x}\AgdaSpace{}%
\AgdaBound{x₁}\AgdaSymbol{))}\<%
\\
\>[0]\AgdaSymbol{(}\AgdaFunction{ii}\AgdaSymbol{)}\AgdaSpace{}%
\AgdaSymbol{=}\AgdaSpace{}%
\AgdaRecord{Σ}\AgdaSpace{}%
\AgdaBound{e}\AgdaSpace{}%
\AgdaSymbol{(λ}\AgdaSpace{}%
\AgdaBound{x}\AgdaSpace{}%
\AgdaSymbol{→}\AgdaSpace{}%
\AgdaRecord{Σ}\AgdaSpace{}%
\AgdaSymbol{(}\AgdaPostulate{woman}\AgdaSpace{}%
\AgdaBound{x}\AgdaSymbol{)}\AgdaSpace{}%
\AgdaSymbol{(λ}\AgdaSpace{}%
\AgdaBound{x₁}\AgdaSpace{}%
\AgdaSymbol{→}\AgdaSpace{}%
\AgdaSymbol{(}\AgdaBound{x₂}\AgdaSpace{}%
\AgdaSymbol{:}\AgdaSpace{}%
\AgdaBound{e}\AgdaSymbol{)}\AgdaSpace{}%
\AgdaSymbol{→}\AgdaSpace{}%
\AgdaPostulate{man}\AgdaSpace{}%
\AgdaBound{x₂}\AgdaSpace{}%
\AgdaSymbol{→}\AgdaSpace{}%
\AgdaPostulate{love}\AgdaSpace{}%
\AgdaBound{x}\AgdaSpace{}%
\AgdaBound{x₂}\AgdaSymbol{))}\<%
\end{code}
\begin{code}[hide]%
\>[0]\AgdaFunction{someManLovesAWoman}\AgdaSpace{}%
\AgdaSymbol{=}\AgdaSpace{}%
\AgdaFunction{sentence}\AgdaSpace{}%
\AgdaSymbol{(}\AgdaFunction{A}\AgdaSpace{}%
\AgdaPostulate{man}\AgdaSymbol{)}\AgdaSpace{}%
\AgdaSymbol{(}\AgdaFunction{verbp}\AgdaSpace{}%
\AgdaFunction{loves}\AgdaSpace{}%
\AgdaSymbol{(}\AgdaFunction{A}\AgdaSpace{}%
\AgdaPostulate{woman}\AgdaSymbol{))}\<%
\\
\>[0]\AgdaFunction{someManLovesAWoman'}\AgdaSpace{}%
\AgdaSymbol{=}\AgdaSpace{}%
\AgdaFunction{sentence}\AgdaSpace{}%
\AgdaSymbol{(}\AgdaFunction{A}\AgdaSpace{}%
\AgdaPostulate{man}\AgdaSymbol{)}\AgdaSpace{}%
\AgdaSymbol{(}\AgdaFunction{verbp}\AgdaSpace{}%
\AgdaFunction{loves'}\AgdaSpace{}%
\AgdaSymbol{(}\AgdaFunction{A}\AgdaSpace{}%
\AgdaPostulate{woman}\AgdaSymbol{))}\<%
\\
\>[0]\AgdaFunction{johnLovesAWoman}\AgdaSpace{}%
\AgdaSymbol{=}\AgdaSpace{}%
\AgdaFunction{sentence}\AgdaSpace{}%
\AgdaFunction{John}\AgdaSpace{}%
\AgdaSymbol{(}\AgdaFunction{verbp}\AgdaSpace{}%
\AgdaFunction{loves}\AgdaSpace{}%
\AgdaSymbol{(}\AgdaFunction{A}\AgdaSpace{}%
\AgdaPostulate{woman}\AgdaSymbol{))}\<%
\\
\>[0]\AgdaFunction{johnLovesAWoman'}\AgdaSpace{}%
\AgdaSymbol{=}\AgdaSpace{}%
\AgdaFunction{sentence}\AgdaSpace{}%
\AgdaFunction{John}\AgdaSpace{}%
\AgdaSymbol{(}\AgdaFunction{verbp}\AgdaSpace{}%
\AgdaFunction{loves'}\AgdaSpace{}%
\AgdaSymbol{(}\AgdaFunction{A}\AgdaSpace{}%
\AgdaPostulate{woman}\AgdaSymbol{))}\<%
\end{code}

%\paragraph{Agda Fracas Example}

Let's articulate natural language inference to a hypothetical mathematician.

\textbf{Theorem}:
  If every man loves a woman, then John loves a woman.

\textbf{Proof}:
Assume that John is a person who is human. Knowing that every man
  loves a woman, we can apply this knowledge to John, who is a man, and identity a
woman that John loves. Specifically, this woman John loves is a person, evidence
that person is a woman, and evidence that John loves that person. {QED}

We also show this argument in Agda, noting that we comment out the more verbose
but informative information about who the woman John loves really is. We
explicitly include the data of the proof commented out so-as to see how the
overly elaborate natural language argument manifests in code.

\begin{code}%
\>[0]\AgdaComment{-- (i)}\<%
\\
\>[0]\AgdaFunction{jlaw}\AgdaSpace{}%
\AgdaSymbol{:}\AgdaSpace{}%
\AgdaFunction{everyManLovesAWoman}\AgdaSpace{}%
\AgdaSymbol{→}\AgdaSpace{}%
\AgdaFunction{johnLovesAWoman}\<%
\\
\>[0]\AgdaFunction{jlaw}\AgdaSpace{}%
\AgdaBound{emlaw}\AgdaSpace{}%
\AgdaSymbol{=}\AgdaSpace{}%
\AgdaFunction{womanJonLoves}\AgdaSpace{}%
\AgdaComment{-- thePerson ,  thePersonIsWoman  , jonLovesThePerson}\<%
\\
\>[0][@{}l@{\AgdaIndent{0}}]%
\>[2]\AgdaKeyword{where}\<%
\\
\>[2][@{}l@{\AgdaIndent{0}}]%
\>[4]\AgdaFunction{womanJonLoves}\AgdaSpace{}%
\AgdaSymbol{:}\AgdaSpace{}%
\AgdaRecord{Σ}\AgdaSpace{}%
\AgdaBound{e}\AgdaSpace{}%
\AgdaSymbol{(λ}\AgdaSpace{}%
\AgdaBound{z}\AgdaSpace{}%
\AgdaSymbol{→}\AgdaSpace{}%
\AgdaRecord{Σ}\AgdaSpace{}%
\AgdaSymbol{(}\AgdaPostulate{woman}\AgdaSpace{}%
\AgdaBound{z}\AgdaSymbol{)}\AgdaSpace{}%
\AgdaSymbol{(λ}\AgdaSpace{}%
\AgdaBound{\AgdaUnderscore{}}\AgdaSpace{}%
\AgdaSymbol{→}\AgdaSpace{}%
\AgdaPostulate{love}\AgdaSpace{}%
\AgdaPostulate{j}\AgdaSpace{}%
\AgdaBound{z}\AgdaSymbol{))}\<%
\\
%
\>[4]\AgdaFunction{womanJonLoves}\AgdaSpace{}%
\AgdaSymbol{=}\AgdaSpace{}%
\AgdaBound{emlaw}\AgdaSpace{}%
\AgdaPostulate{j}\AgdaSpace{}%
\AgdaPostulate{johnMan}\<%
\\
%
\>[4]\AgdaComment{-- thePerson : e}\<%
\\
%
\>[4]\AgdaComment{-- thePerson = proj₁ womanJonLoves}\<%
\\
%
\>[4]\AgdaComment{-- thePersonIsWoman : woman (thePerson)}\<%
\\
%
\>[4]\AgdaComment{-- thePersonIsWoman = proj₁ (proj₂ womanJonLoves)}\<%
\\
%
\>[4]\AgdaComment{-- jonLovesThePerson : love j thePerson}\<%
\\
%
\>[4]\AgdaComment{-- jonLovesThePerson = proj₂ (proj₂ womanJonLoves)}\<%
\end{code}

We note this is the first interpretation of \term{loves}. In the alternative
presentation we see that if there is a woman every man loves, that our semantic
theory interprets her as a person, the evidence they are a woman, and the
evidence that every man loves them. Since every man loves them, John certainly
does as well.

\begin{code}%
\>[0]\AgdaComment{-- (ii)}\<%
\\
\>[0]\AgdaFunction{jlaw'}\AgdaSpace{}%
\AgdaSymbol{:}\AgdaSpace{}%
\AgdaFunction{everyManLovesAWoman'}\AgdaSpace{}%
\AgdaSymbol{→}\AgdaSpace{}%
\AgdaFunction{johnLovesAWoman'}\<%
\\
\>[0]\AgdaFunction{jlaw'}\AgdaSpace{}%
\AgdaSymbol{(}\AgdaBound{person}\AgdaSpace{}%
\AgdaOperator{\AgdaInductiveConstructor{,}}\AgdaSpace{}%
\AgdaBound{personIsWoman}\AgdaSpace{}%
\AgdaOperator{\AgdaInductiveConstructor{,}}\AgdaSpace{}%
\AgdaBound{everyManLovesPerson}\AgdaSpace{}%
\AgdaSymbol{)}\AgdaSpace{}%
\AgdaSymbol{=}\<%
\\
\>[0][@{}l@{\AgdaIndent{0}}]%
\>[2]\AgdaBound{person}\AgdaSpace{}%
\AgdaOperator{\AgdaInductiveConstructor{,}}\AgdaSpace{}%
\AgdaBound{personIsWoman}\AgdaSpace{}%
\AgdaOperator{\AgdaInductiveConstructor{,}}\AgdaSpace{}%
\AgdaBound{everyManLovesPerson}\AgdaSpace{}%
\AgdaPostulate{j}\AgdaSpace{}%
\AgdaPostulate{johnMan}\<%
\end{code}

Finally, we can know that some man loves a woman by virtue of the fact that John
loves a woman and John is a man. We note that the proofs \term{smlw} and
\term{smlw'} just articulate the data in different order.

\begin{code}%
\>[0]\AgdaFunction{smlw}\AgdaSpace{}%
\AgdaSymbol{:}\AgdaSpace{}%
\AgdaFunction{johnLovesAWoman}\AgdaSpace{}%
\AgdaSymbol{→}\AgdaSpace{}%
\AgdaFunction{someManLovesAWoman}\<%
\\
\>[0]\AgdaFunction{smlw}\AgdaSpace{}%
\AgdaSymbol{(}\AgdaBound{w}\AgdaSpace{}%
\AgdaOperator{\AgdaInductiveConstructor{,}}\AgdaSpace{}%
\AgdaBound{wWoman}\AgdaSpace{}%
\AgdaOperator{\AgdaInductiveConstructor{,}}\AgdaSpace{}%
\AgdaBound{jlovesw}\AgdaSpace{}%
\AgdaSymbol{)}\AgdaSpace{}%
\AgdaSymbol{=}\AgdaSpace{}%
\AgdaPostulate{j}\AgdaSpace{}%
\AgdaOperator{\AgdaInductiveConstructor{,}}\AgdaSpace{}%
\AgdaPostulate{johnMan}\AgdaSpace{}%
\AgdaOperator{\AgdaInductiveConstructor{,}}\AgdaSpace{}%
\AgdaBound{w}\AgdaSpace{}%
\AgdaOperator{\AgdaInductiveConstructor{,}}\AgdaSpace{}%
\AgdaBound{wWoman}\AgdaSpace{}%
\AgdaOperator{\AgdaInductiveConstructor{,}}\AgdaSpace{}%
\AgdaBound{jlovesw}\<%
\\
\>[0]\AgdaFunction{smlw'}\AgdaSpace{}%
\AgdaSymbol{:}\AgdaSpace{}%
\AgdaFunction{johnLovesAWoman'}\AgdaSpace{}%
\AgdaSymbol{→}\AgdaSpace{}%
\AgdaFunction{someManLovesAWoman'}\<%
\\
\>[0]\AgdaFunction{smlw'}\AgdaSpace{}%
\AgdaSymbol{(}\AgdaBound{w}\AgdaSpace{}%
\AgdaOperator{\AgdaInductiveConstructor{,}}\AgdaSpace{}%
\AgdaBound{wWoman}\AgdaSpace{}%
\AgdaOperator{\AgdaInductiveConstructor{,}}\AgdaSpace{}%
\AgdaBound{jlovesw}\AgdaSpace{}%
\AgdaSymbol{)}\AgdaSpace{}%
\AgdaSymbol{=}\AgdaSpace{}%
\AgdaBound{w}\AgdaSpace{}%
\AgdaOperator{\AgdaInductiveConstructor{,}}\AgdaSpace{}%
\AgdaBound{wWoman}\AgdaSpace{}%
\AgdaOperator{\AgdaInductiveConstructor{,}}\AgdaSpace{}%
\AgdaPostulate{j}\AgdaSpace{}%
\AgdaOperator{\AgdaInductiveConstructor{,}}\AgdaSpace{}%
\AgdaPostulate{johnMan}\AgdaSpace{}%
\AgdaOperator{\AgdaInductiveConstructor{,}}\AgdaSpace{}%
\AgdaBound{jlovesw}\<%
\end{code}

While the Montagovian approach is historically interesting and still incredibly
significant in linguistic semantics, the interpretation of various parts of
speech and their means of syntactic combination doesn't always seem to be intuitively reflected in the types. Dependent type theories, or MTTs, which have a more expressive type system, gives us a means of more intuitively encoding the meanings.

This use of a ``fancier" type theory for NL semantics can be viewed as analogous
to a preference of inductively defined numbers over Von Neumman or Chruch
encodings. While the latter constructions are interesting and important
historically, they aren't easy to work with and don't match our intuition.
Nonetheless, the different encodings of numbers in different foundations can be
proven sound and complete with respect to each other, so that one can rest
assured that the intuitive notion of number is captured by all of them.

Unlike in formal mathematics, however, to capturing semantics in different
foundational theories suggests no way of proving soundness or completeness with
respect to the interpretation of different phrases, as the set of semantically
fluent utterances are not inductively defined. One may instead take a pragmatic
approach and see the difficulties arising when doing inference with respect to
different type theories.
