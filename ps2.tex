\documentclass[a4paper, 11pt]{article}

\usepackage[ mincrossrefs=999, style=numeric, backend=biber, url=false,
isbn=false, doi=false, ]{biblatex}

\addbibresource{references.bib}

\usepackage[margin=1in]{geometry} \usepackage[dvipsnames]{xcolor}
\usepackage[colorlinks]{hyperref} \usepackage{enumitem} \usepackage{amsfonts}

\usepackage{unicode-math}
\usepackage{stmaryrd}
\usepackage{amsfonts}
\usepackage{mathtools}
\usepackage{xspace}

\usepackage{newunicodechar}
\newunicodechar{ℓ}{\ensuremath{\mathnormal\ell}}
\newunicodechar{→}{\ensuremath{\mathnormal\rightarrow}}
\newunicodechar{∈}{\ensuremath{\mathnormal\in}}
\newunicodechar{λ}{$\lambda$}

\newunicodechar{⌞}{$\llcorner$}
\newunicodechar{⌟}{$\lrcorner$}
\newunicodechar{⊚}{$\circledcirc$}
\newunicodechar{⊤}{$\top$}
\newunicodechar{⊥}{$\perp$}


\NewDocumentCommand{\codeword}{v}{%
\texttt{\textcolor{gray}{#1}}%
}

\NewDocumentCommand{\term}{v}{%
\texttt{\textcolor{blue}{#1}}%
}
\NewDocumentCommand{\keyword}{v}{%
\texttt{\textcolor{orange}{#1}}%
}



\usepackage{ stmaryrd }
\usepackage{agda}


% \usepackage{enumitem}
\setlist[itemize]{noitemsep, topsep=0pt}

\hypersetup{ citecolor=RoyalBlue }

\usepackage{fontspec}

\usepackage{titlesec}

\titlespacing\section{0pt}{4pt plus 2pt minus 2pt}{4pt plus 2pt minus 2pt}

% \setmainfont{Linux Libertine} % \setsansfont{Linux Biolinum} % %
% \setmonofont[Scale=0.85]{PragmataPro Mono Liga}

\begin{document} \pagenumbering{gobble}

\begin{titlepage}

\vspace*{1cm}

\begin{center} \Large Report for Type Theory and Natural Language Semantics\\ 

\vspace{1.5cm}

\large Warrick Scott Macmillan \end{center}

\end{titlepage}

\section{Introduction} 

Since Richard Montague's seminal work investigating the natural language (NL)
semantics of quantifiers via typed, intentional Higher Order Logic (HOL)
\cite{Montague1973}, the have been many subsequent iterations. These include:

\begin{itemize}
\item How to interface various syntactic grammar formalisms with semantic theories 
\item How to leverage different logics and type theories to model natural
  lanaguage semantics
\item How to create a systems that one can use to empirically test 
  semantic theories on real data
\end{itemize}

Montague, a student of Alfred Tarski, was working in the \emph{model-theoretic}
tradition of logic. The \emph{proof-theoretic} tradition of logic, beginning
with Gentzen \cite{Gentzen1935} and continued by Pragwitz
\cite{prawitz2006natural}, led to the critical developments of Per Martin-Löf's
investigations of a constructive foundations of mathematics \cite{ml79}
\cite{ml1984}. Martin-Löf Type Theory (MLTT) was applied to natural language
semantics after a discussion between Per and Göran Sundholm about the infamously
tricky \emph{donkey anaphora} \cite{Sundholm1986}.

Soon thereafter Martin-Löf's student, Aarne Ranta, developed the full theory
which applied MLTT to understand natural language semantics in a proof theoretic
tradition, tradition very much inspired by but divergent from Montague
\cite{ranta1994type}. While Ranta's research focus shifted largely from
semantics to syntax via his occupation with developing the programming language
Grammatical Framework (GF) \cite{gf}, his original semantic work greatly
influenced both linguists and computer scientists. Luo, a type theorist whose
early work was an iteration of MLTT \cite{luobook} , was one of Ranta's primary
successors in this endeavor, and along with linguists like Stergios, there has
been much interest elaborating and expanding Ranta's original seed
\cite{cnTypes} \cite{luoSterg}. It has been articulated that both
the proof and model theoretic approaches to logic cohere in Modern Type
Theoeries (MTTs) and their application in NL semantics \cite{luoMt}.

One of the most central ideas in type theoretic semantics in contrast to those
in Montagovian tradition, is the ``common nouns as types" maxim \cite{cnTypes},
whereby the common nouns are actually a universe, instead of functions in the
classical logic setting. This not only fits a more natural intuition, but also
makes it convenient for creating an elaborate subtyping mechanism, namely 
coercive subtyping as developed by Luo \cite{luoCoer} \cite{luo13}.

Despite the obstacles that subtyping presents in that it disallows uniqueness of
typing, the coercive subtyping approach allows one to retain nice
meta-properties about the type theory like canonicity, while allowing one to
construct an ontological hierarchy that captures semantic nuance and facilitates
computation. Computation using coercive types is just one of the many benefits
one can leverage from this MTT approach to linguistic semantics.

Proof assistants like Coq and Agda are implementations of different dependent
type theories. They allow one to interactively build proofs, or programs, which
are implemented according to their specified behaviors, or types. Because the
types are identified with theorems by way of the propositions-as-types paradigm,
the proof assistants are capable of doing functional programming, advanced
program verification, and constructive mathematics. It is possible to shallowly
embed semantic encodings from both the Montagovian and MTT traditions in these
proof assistants as was done in Coq \cite{luoCoq} \cite{fracoq}.

The dependent function type is core to the type theories, and it is possible to
prove implications by constructing functions. One can therefore do inference
about the semantic encodings. Inference is one important way of empirically
testing or observing a semantic theories' success. The FraCas test suite
\cite{cooper1996using} was designed to capture the inferability of various
semantic phenomena in a suite of 346 question, each of which has at least one
premise from which a native speaker would be able to affirm, deny, or defer the
question if an answer is not knowable under the assumptions.

\section{A Montagovian Example}

We here showcase an example similar to the FraCas test suite to demonstrate the
way in which one makes inference with Agda. This takes place after having
interpreted the syntax and constructed semantic formulas in
Montague's type theory.

\begin{verbatim}
Premise    : Every man loves a woman.
Question 1 : Does John love a woman?
Answer 1 : Yes.
Question 2 : Does some man love a woman?
Answer 2   : Yes.
\end{verbatim}

\subsubsection{Montague}

We can, given a means of constructing trees out of basic syntactic categories,
assign types to the categories and functions to the rules which obey the rules'
signatures. One can decide, based on some GF abstract syntax, how these GF
functions evaluate to formulas in the logic. This then allows one to derive
meaningful sentences based off the abstract syntax trees, by normalizing the
lambda terms. The failure of grammatical terms to normalize (enough) is a sign
of semantic incoherence. This can either be the result of improper typing
assignments for given lexical categories or a failure to give ``proper" lambda
terms to given rules or lexical constants. It is also a goal of the theory to
only admit semantically reasonable ideas, i.e. that there aren't superfluous,
meaningful sentences which evaluate.

While FraCoq used trees straight generated from the GF Resource Grammar Library
\cite{ljunglof2012bilingual}, we choose here a simpler syntax taken from
\cite{nassli}. Our implementation differs from Fracoq in a few ways. First, we
choose different type assignments for the grammatical categories. Additionally,
we use Agda instead of Coq.

In the Montagovian tradition one uses a simple type theory. There are two
basic types, entities and formulas, denoted $e$ and $t$, respectively.
Everything else is constructed as of higher order functions ending in $t$. The
entities are meant to represent some notion of objective thing in the world,
like John, whereas the formulas, only occurring in the codomain of a function,
may represent utterances such as ``John walks".

Given a suite with nouns (N), verbs (V), noun phrases (NP), verb phrases (VP),
determiners (Det) and sentences (S) we can then give a Montagovian
interpretation to accommodate the above FraCas-like example with the following.
We assign the grammatical categories, and functions over those categories, as an
abstract syntax in GF:

\begin{verbatim}
cat
  S ; N ; NP ; V; VP ; V ; Det ;
fun
  sentence : NP -> VP -> S ;
  verbp : V -> NP -> VP ;
  ...
\end{verbatim}

As was done with Coq in \cite{fracoq}, we can embed such a grammar in Agda.
First, we assign GF categories to Agda sets.

$$\llbracket\_\rrbracket\; {:}\; Cat_{GF} \rightarrow Set_{Agda}$$

For the functions in the abstract syntax, we simply map the interpretation
functorially with respect to the arrows.

\begin{align*}
  \llbracket\_\rrbracket\; {:}\; fun_{Cat} &\longrightarrow fun_{Agda}\\
  A \rightarrow B &\mapsto \llbracket A \rrbracket \rightarrow \llbracket B \rrbracket
\end{align*}

We then know that the interpretation a given GF function $f {:} X$, must be well
typed with the Agda semantics, i.e. $\llbracket f \rrbracket {:} \llbracket X
\rrbracket$. Finally, the way one constructs ASTs, by plugging in functions (or
leaves) with the correct return type into a given function (or node), and
evaluating based of applying the function at a node to its leaves in successive
order. These become Agda function applications:

$$\llbracket f(g) \rrbracket \rightarrow \llbracket f \rrbracket (\llbracket g \rrbracket)$$

\begin{code}[hide]%
\>[0]\AgdaKeyword{module}%
\>[0I]\AgdaModule{MS}\<%
\\
\>[0I][@{}l@{\AgdaIndent{0}}]%
\>[10]\AgdaSymbol{(}\AgdaBound{e}\AgdaSpace{}%
\AgdaSymbol{:}\AgdaSpace{}%
\AgdaPrimitive{Set}\AgdaSymbol{)}\AgdaSpace{}%
\AgdaKeyword{where}\<%
\\
\>[0]\AgdaKeyword{open}\AgdaSpace{}%
\AgdaKeyword{import}\AgdaSpace{}%
\AgdaModule{Data.Product}\AgdaSpace{}%
\AgdaKeyword{using}\AgdaSpace{}%
\AgdaSymbol{(}\AgdaRecord{Σ}\AgdaSymbol{;}\AgdaSpace{}%
\AgdaOperator{\AgdaFunction{\AgdaUnderscore{}×\AgdaUnderscore{}}}\AgdaSymbol{;}\AgdaSpace{}%
\AgdaOperator{\AgdaInductiveConstructor{\AgdaUnderscore{},\AgdaUnderscore{}}}\AgdaSymbol{;}\AgdaSpace{}%
\AgdaField{proj₁}\AgdaSymbol{;}\AgdaSpace{}%
\AgdaField{proj₂}\AgdaSymbol{;}\AgdaSpace{}%
\AgdaFunction{∃}\AgdaSymbol{;}\AgdaSpace{}%
\AgdaFunction{Σ-syntax}\AgdaSymbol{;}\AgdaSpace{}%
\AgdaFunction{∃-syntax}\AgdaSymbol{)}\<%
\\
\>[0]\AgdaComment{-- example from}\<%
\\
\>[0]\AgdaComment{-- https://pdfs.semanticscholar.org/f94b/268c1c91dd1de22cf978a7ea03f8860cbe9d.pdf}\<%
\end{code}

We assign Agda types to the categories, knowing that entites are postulated as a
type.

\begin{code}%
\>[0]\AgdaFunction{t}\AgdaSpace{}%
\AgdaSymbol{=}\AgdaSpace{}%
\AgdaPrimitive{Set}\<%
\\
\>[0]\AgdaFunction{S}\AgdaSpace{}%
\AgdaSymbol{=}\AgdaSpace{}%
\AgdaFunction{t}\<%
\\
\>[0]\AgdaFunction{NP}\AgdaSpace{}%
\AgdaSymbol{=}\AgdaSpace{}%
\AgdaSymbol{(}\AgdaBound{e}\AgdaSpace{}%
\AgdaSymbol{→}\AgdaSpace{}%
\AgdaFunction{t}\AgdaSymbol{)}\AgdaSpace{}%
\AgdaSymbol{→}\AgdaSpace{}%
\AgdaFunction{t}\<%
\\
\>[0]\AgdaFunction{VP}\AgdaSpace{}%
\AgdaSymbol{=}\AgdaSpace{}%
\AgdaFunction{NP}\AgdaSpace{}%
\AgdaSymbol{→}\AgdaSpace{}%
\AgdaFunction{t}\<%
\\
\>[0]\AgdaFunction{V}\AgdaSpace{}%
\AgdaSymbol{=}\AgdaSpace{}%
\AgdaFunction{NP}\AgdaSpace{}%
\AgdaSymbol{→}\AgdaSpace{}%
\AgdaFunction{NP}\AgdaSpace{}%
\AgdaSymbol{→}\AgdaSpace{}%
\AgdaFunction{t}\<%
\\
\>[0]\AgdaFunction{Det}\AgdaSpace{}%
\AgdaSymbol{=}\AgdaSpace{}%
\AgdaSymbol{(}\AgdaBound{e}\AgdaSpace{}%
\AgdaSymbol{→}\AgdaSpace{}%
\AgdaFunction{t}\AgdaSymbol{)}\AgdaSpace{}%
\AgdaSymbol{→}\AgdaSpace{}%
\AgdaSymbol{(}\AgdaBound{e}\AgdaSpace{}%
\AgdaSymbol{→}\AgdaSpace{}%
\AgdaFunction{t}\AgdaSymbol{)}\AgdaSpace{}%
\AgdaSymbol{→}\AgdaSpace{}%
\AgdaFunction{t}\<%
\\
\>[0]\AgdaFunction{N}\AgdaSpace{}%
\AgdaSymbol{=}\AgdaSpace{}%
\AgdaBound{e}\AgdaSpace{}%
\AgdaSymbol{→}\AgdaSpace{}%
\AgdaFunction{t}\<%
\end{code}

Agda's equality symbol can be seen as the semantic interpretation,
$\llbracket_\rrbracket$, of our grammar's categories, whereby we are here
working with a shallow embedding. One could instead elect to define both the GF
embedding as a record and Montague's intentional type theory as an inductive
dataype, whereby the semantics could then be given explictly. This degree of
formality is not necessary for our simple examples, but doing so would allow one
to prove metaproperties about the GF embedding, and should certainly be
investigated.
The two GF functions for forming sentences and verb phrases can
then be given the following Agda interpreations.

\begin{code}%
\>[0]\AgdaFunction{sentence}\AgdaSpace{}%
\AgdaSymbol{:}\AgdaSpace{}%
\AgdaFunction{NP}\AgdaSpace{}%
\AgdaSymbol{→}\AgdaSpace{}%
\AgdaFunction{VP}\AgdaSpace{}%
\AgdaSymbol{→}\AgdaSpace{}%
\AgdaFunction{S}\<%
\\
\>[0]\AgdaFunction{sentence}\AgdaSpace{}%
\AgdaBound{S}\AgdaSpace{}%
\AgdaBound{V}\AgdaSpace{}%
\AgdaSymbol{=}\AgdaSpace{}%
\AgdaBound{V}\AgdaSpace{}%
\AgdaBound{S}\<%
\\
%
\\[\AgdaEmptyExtraSkip]%
\>[0]\AgdaFunction{verbp}\AgdaSpace{}%
\AgdaSymbol{:}\AgdaSpace{}%
\AgdaFunction{V}\AgdaSpace{}%
\AgdaSymbol{→}\AgdaSpace{}%
\AgdaFunction{NP}\AgdaSpace{}%
\AgdaSymbol{→}\AgdaSpace{}%
\AgdaFunction{VP}\<%
\\
\>[0]\AgdaFunction{verbp}\AgdaSpace{}%
\AgdaBound{V}\AgdaSpace{}%
\AgdaBound{O}\AgdaSpace{}%
\AgdaBound{S}\AgdaSpace{}%
\AgdaSymbol{=}\AgdaSpace{}%
\AgdaBound{V}\AgdaSpace{}%
\AgdaBound{S}\AgdaSpace{}%
\AgdaBound{O}\<%
\end{code}

\begin{code}[hide]%
\>[0]\AgdaFunction{nounp}\AgdaSpace{}%
\AgdaSymbol{:}\AgdaSpace{}%
\AgdaFunction{Det}\AgdaSpace{}%
\AgdaSymbol{→}\AgdaSpace{}%
\AgdaFunction{N}\AgdaSpace{}%
\AgdaSymbol{→}\AgdaSpace{}%
\AgdaFunction{NP}\<%
\\
\>[0]\AgdaFunction{nounp}\AgdaSpace{}%
\AgdaBound{D}\AgdaSpace{}%
\AgdaBound{N}\AgdaSpace{}%
\AgdaSymbol{=}\AgdaSpace{}%
\AgdaBound{D}\AgdaSpace{}%
\AgdaBound{N}\<%
\end{code}

We axiomatically include primitive lexical items of \term{love}, an entity
\term{j} for John, and \term{man} and \term{woman} as nouns, so that we can
articulate logically interesting facts about them. As our encoding of noun
phrases takes an arguement, we can define the syntactic notion \term{John} by
applying an arguement function to \term{j}.

\begin{code}%
\>[0]\AgdaKeyword{postulate}\<%
\\
\>[0][@{}l@{\AgdaIndent{0}}]%
\>[2]\AgdaPostulate{love}\AgdaSpace{}%
\AgdaSymbol{:}\AgdaSpace{}%
\AgdaBound{e}\AgdaSpace{}%
\AgdaSymbol{→}\AgdaSpace{}%
\AgdaBound{e}\AgdaSpace{}%
\AgdaSymbol{→}\AgdaSpace{}%
\AgdaFunction{t}\<%
\\
%
\>[2]\AgdaPostulate{j}\AgdaSpace{}%
\AgdaSymbol{:}\AgdaSpace{}%
\AgdaBound{e}\<%
\\
%
\>[2]\AgdaPostulate{man}\AgdaSpace{}%
\AgdaSymbol{:}\AgdaSpace{}%
\AgdaFunction{N}\<%
\\
%
\>[2]\AgdaPostulate{woman}\AgdaSpace{}%
\AgdaSymbol{:}\AgdaSpace{}%
\AgdaFunction{N}\<%
\\
%
\>[2]\AgdaPostulate{johnMan}\AgdaSpace{}%
\AgdaSymbol{:}\AgdaSpace{}%
\AgdaPostulate{man}\AgdaSpace{}%
\AgdaPostulate{j}\<%
\\
%
\\[\AgdaEmptyExtraSkip]%
\>[0]\AgdaFunction{John}\AgdaSpace{}%
\AgdaSymbol{:}\AgdaSpace{}%
\AgdaFunction{NP}\<%
\\
\>[0]\AgdaFunction{John}\AgdaSpace{}%
\AgdaBound{P}\AgdaSpace{}%
\AgdaSymbol{=}\AgdaSpace{}%
\AgdaBound{P}\AgdaSpace{}%
\AgdaPostulate{j}\<%
\end{code}

We define our quantifiers using the Agda's dependent $\Pi$ and $\Sigma$ which
make up the core of any dependently typed language.

\begin{code}%
\>[0]\AgdaFunction{Every}\AgdaSpace{}%
\AgdaSymbol{:}\AgdaSpace{}%
\AgdaFunction{Det}\<%
\\
\>[0]\AgdaFunction{Every}\AgdaSpace{}%
\AgdaBound{P}\AgdaSpace{}%
\AgdaBound{Q}\AgdaSpace{}%
\AgdaSymbol{=}\AgdaSpace{}%
\AgdaSymbol{(}\AgdaBound{x}\AgdaSpace{}%
\AgdaSymbol{:}\AgdaSpace{}%
\AgdaBound{e}\AgdaSymbol{)}\AgdaSpace{}%
\AgdaSymbol{→}\AgdaSpace{}%
\AgdaBound{P}\AgdaSpace{}%
\AgdaBound{x}\AgdaSpace{}%
\AgdaSymbol{→}\AgdaSpace{}%
\AgdaBound{Q}\AgdaSpace{}%
\AgdaBound{x}\<%
\\
%
\\[\AgdaEmptyExtraSkip]%
\>[0]\AgdaFunction{A}\AgdaSpace{}%
\AgdaSymbol{:}\AgdaSpace{}%
\AgdaFunction{Det}\<%
\\
\>[0]\AgdaFunction{A}\AgdaSpace{}%
\AgdaBound{cn}\AgdaSpace{}%
\AgdaBound{vp}\AgdaSpace{}%
\AgdaSymbol{=}\AgdaSpace{}%
\AgdaFunction{Σ[}\AgdaSpace{}%
\AgdaBound{x}\AgdaSpace{}%
\AgdaFunction{∈}\AgdaSpace{}%
\AgdaBound{e}\AgdaSpace{}%
\AgdaFunction{]}\AgdaSpace{}%
\AgdaSymbol{(}\AgdaBound{cn}\AgdaSpace{}%
\AgdaBound{x}\AgdaSpace{}%
\AgdaOperator{\AgdaFunction{×}}\AgdaSpace{}%
\AgdaBound{vp}\AgdaSpace{}%
\AgdaBound{x}\AgdaSymbol{)}\<%
\end{code}

We also notice that we may define the dintransative verb loves two ways, which
are obtained commuting the NP arguements.

\begin{code}%
\>[0]\AgdaFunction{loves}\AgdaSpace{}%
\AgdaFunction{loves'}\AgdaSpace{}%
\AgdaSymbol{:}\AgdaSpace{}%
\AgdaFunction{V}\<%
\\
\>[0]\AgdaFunction{loves}\AgdaSpace{}%
\AgdaBound{O}\AgdaSpace{}%
\AgdaBound{S}\AgdaSpace{}%
\AgdaSymbol{=}\AgdaSpace{}%
\AgdaBound{O}\AgdaSpace{}%
\AgdaSymbol{(λ}\AgdaSpace{}%
\AgdaBound{x}\AgdaSpace{}%
\AgdaSymbol{→}\AgdaSpace{}%
\AgdaBound{S}\AgdaSpace{}%
\AgdaSymbol{λ}\AgdaSpace{}%
\AgdaBound{y}\AgdaSpace{}%
\AgdaSymbol{→}\AgdaSpace{}%
\AgdaPostulate{love}\AgdaSpace{}%
\AgdaBound{x}\AgdaSpace{}%
\AgdaBound{y}\AgdaSymbol{)}%
\>[42]\AgdaComment{-- (i)}\<%
\\
\>[0]\AgdaFunction{loves'}\AgdaSpace{}%
\AgdaBound{O}\AgdaSpace{}%
\AgdaBound{S}\AgdaSpace{}%
\AgdaSymbol{=}\AgdaSpace{}%
\AgdaBound{S}\AgdaSpace{}%
\AgdaSymbol{(λ}\AgdaSpace{}%
\AgdaBound{x}\AgdaSpace{}%
\AgdaSymbol{→}\AgdaSpace{}%
\AgdaBound{O}\AgdaSpace{}%
\AgdaSymbol{λ}\AgdaSpace{}%
\AgdaBound{y}\AgdaSpace{}%
\AgdaSymbol{→}\AgdaSpace{}%
\AgdaPostulate{love}\AgdaSpace{}%
\AgdaBound{x}\AgdaSpace{}%
\AgdaBound{y}\AgdaSymbol{)}%
\>[42]\AgdaComment{-- (ii)}\<%
\end{code}

We can then observe two different semantic interpretations of the phrase ``every
man loves a woman".

\begin{code}%
\>[0]\AgdaFunction{everyManLovesAWoman}\AgdaSpace{}%
\AgdaSymbol{=}\AgdaSpace{}%
\AgdaFunction{sentence}\AgdaSpace{}%
\AgdaSymbol{(}\AgdaFunction{Every}\AgdaSpace{}%
\AgdaPostulate{man}\AgdaSymbol{)}\AgdaSpace{}%
\AgdaSymbol{(}\AgdaFunction{verbp}\AgdaSpace{}%
\AgdaFunction{loves}\AgdaSpace{}%
\AgdaSymbol{(}\AgdaFunction{A}\AgdaSpace{}%
\AgdaPostulate{woman}\AgdaSymbol{))}%
\>[69]\AgdaComment{-- (i)}\<%
\\
\>[0]\AgdaFunction{everyManLovesAWoman'}\AgdaSpace{}%
\AgdaSymbol{=}\AgdaSpace{}%
\AgdaFunction{sentence}\AgdaSpace{}%
\AgdaSymbol{(}\AgdaFunction{Every}\AgdaSpace{}%
\AgdaPostulate{man}\AgdaSymbol{)}\AgdaSpace{}%
\AgdaSymbol{(}\AgdaFunction{verbp}\AgdaSpace{}%
\AgdaFunction{loves'}\AgdaSpace{}%
\AgdaSymbol{(}\AgdaFunction{A}\AgdaSpace{}%
\AgdaPostulate{woman}\AgdaSymbol{))}\AgdaSpace{}%
\AgdaComment{-- (ii)}\<%
\end{code}

The two interpretations of ``loves" which differ as to where the respective
arguements are applied in the program, manifest in the semantic unambiguity of
whether there is one or possibly many women are in the context of consideration.
With Agda's help in normalizing these two types, we can see this ambiguity is
reflected in whether the outermost quantifier is universal or existential. Note
he product has been desugared to a non-dependent Σ-type.

\begin{code}%
\>[0]\AgdaSymbol{(}\AgdaFunction{i}\AgdaSymbol{)}%
\>[5]\AgdaSymbol{=}\AgdaSpace{}%
\AgdaSymbol{(}\AgdaBound{x}\AgdaSpace{}%
\AgdaSymbol{:}\AgdaSpace{}%
\AgdaBound{e}\AgdaSymbol{)}\AgdaSpace{}%
\AgdaSymbol{→}\AgdaSpace{}%
\AgdaPostulate{man}\AgdaSpace{}%
\AgdaBound{x}\AgdaSpace{}%
\AgdaSymbol{→}\AgdaSpace{}%
\AgdaRecord{Σ}\AgdaSpace{}%
\AgdaBound{e}\AgdaSpace{}%
\AgdaSymbol{(λ}\AgdaSpace{}%
\AgdaBound{x₁}\AgdaSpace{}%
\AgdaSymbol{→}\AgdaSpace{}%
\AgdaRecord{Σ}\AgdaSpace{}%
\AgdaSymbol{(}\AgdaPostulate{woman}\AgdaSpace{}%
\AgdaBound{x₁}\AgdaSymbol{)}\AgdaSpace{}%
\AgdaSymbol{(λ}\AgdaSpace{}%
\AgdaBound{x₂}\AgdaSpace{}%
\AgdaSymbol{→}\AgdaSpace{}%
\AgdaPostulate{love}\AgdaSpace{}%
\AgdaBound{x}\AgdaSpace{}%
\AgdaBound{x₁}\AgdaSymbol{))}\<%
\\
\>[0]\AgdaSymbol{(}\AgdaFunction{ii}\AgdaSymbol{)}\AgdaSpace{}%
\AgdaSymbol{=}\AgdaSpace{}%
\AgdaRecord{Σ}\AgdaSpace{}%
\AgdaBound{e}\AgdaSpace{}%
\AgdaSymbol{(λ}\AgdaSpace{}%
\AgdaBound{x}\AgdaSpace{}%
\AgdaSymbol{→}\AgdaSpace{}%
\AgdaRecord{Σ}\AgdaSpace{}%
\AgdaSymbol{(}\AgdaPostulate{woman}\AgdaSpace{}%
\AgdaBound{x}\AgdaSymbol{)}\AgdaSpace{}%
\AgdaSymbol{(λ}\AgdaSpace{}%
\AgdaBound{x₁}\AgdaSpace{}%
\AgdaSymbol{→}\AgdaSpace{}%
\AgdaSymbol{(}\AgdaBound{x₂}\AgdaSpace{}%
\AgdaSymbol{:}\AgdaSpace{}%
\AgdaBound{e}\AgdaSymbol{)}\AgdaSpace{}%
\AgdaSymbol{→}\AgdaSpace{}%
\AgdaPostulate{man}\AgdaSpace{}%
\AgdaBound{x₂}\AgdaSpace{}%
\AgdaSymbol{→}\AgdaSpace{}%
\AgdaPostulate{love}\AgdaSpace{}%
\AgdaBound{x}\AgdaSpace{}%
\AgdaBound{x₂}\AgdaSymbol{))}\<%
\end{code}
\begin{code}[hide]%
\>[0]\AgdaFunction{someManLovesAWoman}\AgdaSpace{}%
\AgdaSymbol{=}\AgdaSpace{}%
\AgdaFunction{sentence}\AgdaSpace{}%
\AgdaSymbol{(}\AgdaFunction{A}\AgdaSpace{}%
\AgdaPostulate{man}\AgdaSymbol{)}\AgdaSpace{}%
\AgdaSymbol{(}\AgdaFunction{verbp}\AgdaSpace{}%
\AgdaFunction{loves}\AgdaSpace{}%
\AgdaSymbol{(}\AgdaFunction{A}\AgdaSpace{}%
\AgdaPostulate{woman}\AgdaSymbol{))}\<%
\\
\>[0]\AgdaFunction{someManLovesAWoman'}\AgdaSpace{}%
\AgdaSymbol{=}\AgdaSpace{}%
\AgdaFunction{sentence}\AgdaSpace{}%
\AgdaSymbol{(}\AgdaFunction{A}\AgdaSpace{}%
\AgdaPostulate{man}\AgdaSymbol{)}\AgdaSpace{}%
\AgdaSymbol{(}\AgdaFunction{verbp}\AgdaSpace{}%
\AgdaFunction{loves'}\AgdaSpace{}%
\AgdaSymbol{(}\AgdaFunction{A}\AgdaSpace{}%
\AgdaPostulate{woman}\AgdaSymbol{))}\<%
\\
\>[0]\AgdaFunction{johnLovesAWoman}\AgdaSpace{}%
\AgdaSymbol{=}\AgdaSpace{}%
\AgdaFunction{sentence}\AgdaSpace{}%
\AgdaFunction{John}\AgdaSpace{}%
\AgdaSymbol{(}\AgdaFunction{verbp}\AgdaSpace{}%
\AgdaFunction{loves}\AgdaSpace{}%
\AgdaSymbol{(}\AgdaFunction{A}\AgdaSpace{}%
\AgdaPostulate{woman}\AgdaSymbol{))}\<%
\\
\>[0]\AgdaFunction{johnLovesAWoman'}\AgdaSpace{}%
\AgdaSymbol{=}\AgdaSpace{}%
\AgdaFunction{sentence}\AgdaSpace{}%
\AgdaFunction{John}\AgdaSpace{}%
\AgdaSymbol{(}\AgdaFunction{verbp}\AgdaSpace{}%
\AgdaFunction{loves'}\AgdaSpace{}%
\AgdaSymbol{(}\AgdaFunction{A}\AgdaSpace{}%
\AgdaPostulate{woman}\AgdaSymbol{))}\<%
\end{code}

%\paragraph{Agda Fracas Example}

Let's articulate natural language inference to a hypothetical mathematician.

\textbf{Theorem}:
  If every man loves a woman, then John loves a woman.

\textbf{Proof}:
Assume that John is a person who is human. Knowing that every man
  loves a woman, we can apply this knowledge to John, who is a man, and identity a
woman that John loves. Specifically, this woman John loves is a person, evidence
that person is a woman, and evidence that John loves that person.

We also show this argument in Agda, noting that we comment out the more verbose
but informative information that goes into who a woman John loves really is. We
explicitly include the data of the proof so as to see how the overly elaborate
natural language arguement manifests in code. 

\begin{code}%
\>[0]\AgdaComment{-- (i)}\<%
\\
\>[0]\AgdaFunction{jlaw}\AgdaSpace{}%
\AgdaSymbol{:}\AgdaSpace{}%
\AgdaFunction{everyManLovesAWoman}\AgdaSpace{}%
\AgdaSymbol{→}\AgdaSpace{}%
\AgdaFunction{johnLovesAWoman}\<%
\\
\>[0]\AgdaFunction{jlaw}\AgdaSpace{}%
\AgdaBound{emlaw}\AgdaSpace{}%
\AgdaSymbol{=}\AgdaSpace{}%
\AgdaFunction{womanJonLoves}\AgdaSpace{}%
\AgdaComment{-- thePerson ,  thePersonIsWoman  , jonLovesThePerson}\<%
\\
\>[0][@{}l@{\AgdaIndent{0}}]%
\>[2]\AgdaKeyword{where}\<%
\\
\>[2][@{}l@{\AgdaIndent{0}}]%
\>[4]\AgdaFunction{womanJonLoves}\AgdaSpace{}%
\AgdaSymbol{:}\AgdaSpace{}%
\AgdaRecord{Σ}\AgdaSpace{}%
\AgdaBound{e}\AgdaSpace{}%
\AgdaSymbol{(λ}\AgdaSpace{}%
\AgdaBound{z}\AgdaSpace{}%
\AgdaSymbol{→}\AgdaSpace{}%
\AgdaRecord{Σ}\AgdaSpace{}%
\AgdaSymbol{(}\AgdaPostulate{woman}\AgdaSpace{}%
\AgdaBound{z}\AgdaSymbol{)}\AgdaSpace{}%
\AgdaSymbol{(λ}\AgdaSpace{}%
\AgdaBound{\AgdaUnderscore{}}\AgdaSpace{}%
\AgdaSymbol{→}\AgdaSpace{}%
\AgdaPostulate{love}\AgdaSpace{}%
\AgdaPostulate{j}\AgdaSpace{}%
\AgdaBound{z}\AgdaSymbol{))}\<%
\\
%
\>[4]\AgdaFunction{womanJonLoves}\AgdaSpace{}%
\AgdaSymbol{=}\AgdaSpace{}%
\AgdaBound{emlaw}\AgdaSpace{}%
\AgdaPostulate{j}\AgdaSpace{}%
\AgdaPostulate{johnMan}\<%
\\
%
\>[4]\AgdaComment{-- thePerson : e}\<%
\\
%
\>[4]\AgdaComment{-- thePerson = proj₁ womanJonLoves}\<%
\\
%
\>[4]\AgdaComment{-- thePersonIsWoman : woman (thePerson)}\<%
\\
%
\>[4]\AgdaComment{-- thePersonIsWoman = proj₁ (proj₂ womanJonLoves)}\<%
\\
%
\>[4]\AgdaComment{-- jonLovesThePerson : love j thePerson}\<%
\\
%
\>[4]\AgdaComment{-- jonLovesThePerson = proj₂ (proj₂ womanJonLoves)}\<%
\end{code}

We note this is the first interpretation of love. In the alternative
presentation we see that if there is a woman every man loves, that our semantic
theory interprets her as a person, the evidence that person is a woman, and the
evidence that every man loves that person. Since every man loves that person,
John certainly does as well.

\begin{code}%
\>[0]\AgdaComment{-- (ii)}\<%
\\
\>[0]\AgdaFunction{jlaw'}\AgdaSpace{}%
\AgdaSymbol{:}\AgdaSpace{}%
\AgdaFunction{everyManLovesAWoman'}\AgdaSpace{}%
\AgdaSymbol{→}\AgdaSpace{}%
\AgdaFunction{johnLovesAWoman'}\<%
\\
\>[0]\AgdaFunction{jlaw'}\AgdaSpace{}%
\AgdaSymbol{(}\AgdaBound{person}\AgdaSpace{}%
\AgdaOperator{\AgdaInductiveConstructor{,}}\AgdaSpace{}%
\AgdaBound{personIsWoman}\AgdaSpace{}%
\AgdaOperator{\AgdaInductiveConstructor{,}}\AgdaSpace{}%
\AgdaBound{everyManLovesPerson}\AgdaSpace{}%
\AgdaSymbol{)}\AgdaSpace{}%
\AgdaSymbol{=}\<%
\\
\>[0][@{}l@{\AgdaIndent{0}}]%
\>[2]\AgdaBound{person}\AgdaSpace{}%
\AgdaOperator{\AgdaInductiveConstructor{,}}\AgdaSpace{}%
\AgdaBound{personIsWoman}\AgdaSpace{}%
\AgdaOperator{\AgdaInductiveConstructor{,}}\AgdaSpace{}%
\AgdaBound{everyManLovesPerson}\AgdaSpace{}%
\AgdaPostulate{j}\AgdaSpace{}%
\AgdaPostulate{johnMan}\<%
\end{code}

Finally, we can know that some man loves a woman by virtue of the fact that John
loves a woman and John is a man. We note that the proofs just articulate the
data in different order.

\begin{code}%
\>[0]\AgdaFunction{smlw}\AgdaSpace{}%
\AgdaSymbol{:}\AgdaSpace{}%
\AgdaFunction{johnLovesAWoman}\AgdaSpace{}%
\AgdaSymbol{→}\AgdaSpace{}%
\AgdaFunction{someManLovesAWoman}\<%
\\
\>[0]\AgdaFunction{smlw}\AgdaSpace{}%
\AgdaSymbol{(}\AgdaBound{w}\AgdaSpace{}%
\AgdaOperator{\AgdaInductiveConstructor{,}}\AgdaSpace{}%
\AgdaBound{wWoman}\AgdaSpace{}%
\AgdaOperator{\AgdaInductiveConstructor{,}}\AgdaSpace{}%
\AgdaBound{jlovesw}\AgdaSpace{}%
\AgdaSymbol{)}\AgdaSpace{}%
\AgdaSymbol{=}\AgdaSpace{}%
\AgdaPostulate{j}\AgdaSpace{}%
\AgdaOperator{\AgdaInductiveConstructor{,}}\AgdaSpace{}%
\AgdaPostulate{johnMan}\AgdaSpace{}%
\AgdaOperator{\AgdaInductiveConstructor{,}}\AgdaSpace{}%
\AgdaBound{w}\AgdaSpace{}%
\AgdaOperator{\AgdaInductiveConstructor{,}}\AgdaSpace{}%
\AgdaBound{wWoman}\AgdaSpace{}%
\AgdaOperator{\AgdaInductiveConstructor{,}}\AgdaSpace{}%
\AgdaBound{jlovesw}\<%
\\
\>[0]\AgdaFunction{smlw'}\AgdaSpace{}%
\AgdaSymbol{:}\AgdaSpace{}%
\AgdaFunction{johnLovesAWoman'}\AgdaSpace{}%
\AgdaSymbol{→}\AgdaSpace{}%
\AgdaFunction{someManLovesAWoman'}\<%
\\
\>[0]\AgdaFunction{smlw'}\AgdaSpace{}%
\AgdaSymbol{(}\AgdaBound{w}\AgdaSpace{}%
\AgdaOperator{\AgdaInductiveConstructor{,}}\AgdaSpace{}%
\AgdaBound{wWoman}\AgdaSpace{}%
\AgdaOperator{\AgdaInductiveConstructor{,}}\AgdaSpace{}%
\AgdaBound{jlovesw}\AgdaSpace{}%
\AgdaSymbol{)}\AgdaSpace{}%
\AgdaSymbol{=}\AgdaSpace{}%
\AgdaBound{w}\AgdaSpace{}%
\AgdaOperator{\AgdaInductiveConstructor{,}}\AgdaSpace{}%
\AgdaBound{wWoman}\AgdaSpace{}%
\AgdaOperator{\AgdaInductiveConstructor{,}}\AgdaSpace{}%
\AgdaPostulate{j}\AgdaSpace{}%
\AgdaOperator{\AgdaInductiveConstructor{,}}\AgdaSpace{}%
\AgdaPostulate{johnMan}\AgdaSpace{}%
\AgdaOperator{\AgdaInductiveConstructor{,}}\AgdaSpace{}%
\AgdaBound{jlovesw}\<%
\end{code}

While the Montagovian approach is historically interesting and still incredibly
significant in linguistic semantics, the interpretation of various parts of
speech and their means of syntactic combination doesn't always seem to be intuitively reflected in the types. Dependent type theories, or MTTs, which have a more expressive type system, gives us a means of more intuitively encoding the meanings.

This use of a ``fancier" type theory for NL semantics can be viewed as analagous
to a preference of inductively defined numbers over Von Neumman or Chruch
encodings. While the latter constructions are interesting and important
historically, they aren't easy to work with and don't match our intuition.
Nonetheless, the different encodings of numbers in different foundations can be
proven sound and complete with respect to each other, so that one can rest
assured that the intuitive notion of number is captured by all of them.

Unlike in formal mathematics, however, to capturing semantics in different
foundational theories suggests no way of proving soundness or completeness with
respect to the interpretation of different phrases, as the set of semantically
fluent utterances are not inductively defined. One may instead take a pragmatic
approach and see the difficulties arising when doing inference with respect to
different type theories.


\section{An MTT Example}

\begin{code}[hide]%
\>[0]\AgdaSymbol{\{-\#}\AgdaSpace{}%
\AgdaKeyword{OPTIONS}\AgdaSpace{}%
\AgdaPragma{--type-in-type}\AgdaSpace{}%
\AgdaSymbol{\#-\}}\<%
\\
%
\\[\AgdaEmptyExtraSkip]%
\>[0]\AgdaKeyword{module}\AgdaSpace{}%
\AgdaModule{Trial}\AgdaSpace{}%
\AgdaKeyword{where}\<%
\\
%
\\[\AgdaEmptyExtraSkip]%
\>[0]\AgdaKeyword{open}\AgdaSpace{}%
\AgdaKeyword{import}\AgdaSpace{}%
\AgdaModule{Data.Product}\AgdaSpace{}%
\AgdaKeyword{using}\AgdaSpace{}%
\AgdaSymbol{(}\AgdaRecord{Σ}\AgdaSymbol{;}\AgdaSpace{}%
\AgdaOperator{\AgdaFunction{\AgdaUnderscore{}×\AgdaUnderscore{}}}\AgdaSymbol{;}\AgdaSpace{}%
\AgdaOperator{\AgdaInductiveConstructor{\AgdaUnderscore{},\AgdaUnderscore{}}}\AgdaSymbol{;}\AgdaSpace{}%
\AgdaField{proj₁}\AgdaSymbol{;}\AgdaSpace{}%
\AgdaField{proj₂}\AgdaSymbol{;}\AgdaSpace{}%
\AgdaFunction{∃}\AgdaSymbol{;}\AgdaSpace{}%
\AgdaFunction{Σ-syntax}\AgdaSymbol{;}\AgdaSpace{}%
\AgdaFunction{∃-syntax}\AgdaSymbol{)}\<%
\\
\>[0]\AgdaKeyword{open}\AgdaSpace{}%
\AgdaKeyword{import}\AgdaSpace{}%
\AgdaModule{Data.Nat}\AgdaSpace{}%
\AgdaKeyword{using}\AgdaSpace{}%
\AgdaSymbol{(}\AgdaDatatype{ℕ}\AgdaSymbol{)}\<%
\\
\>[0]\AgdaKeyword{open}\AgdaSpace{}%
\AgdaKeyword{import}\AgdaSpace{}%
\AgdaModule{Data.Unit}\<%
\\
\>[0]\AgdaKeyword{import}\AgdaSpace{}%
\AgdaModule{Relation.Binary.PropositionalEquality}\AgdaSpace{}%
\AgdaSymbol{as}\AgdaSpace{}%
\AgdaModule{Eq}\<%
\\
\>[0]\AgdaKeyword{open}\AgdaSpace{}%
\AgdaModule{Eq}\AgdaSpace{}%
\AgdaKeyword{using}\AgdaSpace{}%
\AgdaSymbol{(}\AgdaOperator{\AgdaDatatype{\AgdaUnderscore{}≡\AgdaUnderscore{}}}\AgdaSymbol{;}\AgdaSpace{}%
\AgdaInductiveConstructor{refl}\AgdaSymbol{;}\AgdaSpace{}%
\AgdaFunction{trans}\AgdaSymbol{;}\AgdaSpace{}%
\AgdaFunction{sym}\AgdaSymbol{;}\AgdaSpace{}%
\AgdaFunction{cong}\AgdaSymbol{;}\AgdaSpace{}%
\AgdaFunction{cong-app}\AgdaSymbol{;}\AgdaSpace{}%
\AgdaFunction{subst}\AgdaSymbol{)}\<%
\\
\>[0]\AgdaKeyword{open}\AgdaSpace{}%
\AgdaModule{Eq.≡-Reasoning}\AgdaSpace{}%
\AgdaKeyword{using}\AgdaSpace{}%
\AgdaSymbol{(}\AgdaOperator{\AgdaFunction{begin\AgdaUnderscore{}}}\AgdaSymbol{;}\AgdaSpace{}%
\AgdaOperator{\AgdaFunction{\AgdaUnderscore{}≡⟨⟩\AgdaUnderscore{}}}\AgdaSymbol{;}\AgdaSpace{}%
\AgdaFunction{step-≡}\AgdaSymbol{;}\AgdaSpace{}%
\AgdaOperator{\AgdaFunction{\AgdaUnderscore{}∎}}\AgdaSymbol{)}\<%
\end{code}

In this example, we follow the dependently typed approach initiated by Ranta to
do inference on actual FraCas examples.

Initially, one takes the common nouns as types literally, by saying that the
type of common nouns is actually just a universe, which simply gives the
universe an alias of \term{CN} in Agda, $\llbracket CN \rrbracket := Set$. A man
is common noun, so semantically we just say $\llbracket Man \rrbracket\; {:}\;
\llbracket CN \rrbracket$. And if there is a man John, we simply assert
$\llbracket John \rrbracket\; {:}\; \llbracket Man \rrbracket$.

\begin{code}%
\>[0]\AgdaFunction{CN}\AgdaSpace{}%
\AgdaSymbol{=}\AgdaSpace{}%
\AgdaPrimitive{Set}\<%
\\
%
\\[\AgdaEmptyExtraSkip]%
\>[0]\AgdaKeyword{postulate}\<%
\\
\>[0][@{}l@{\AgdaIndent{0}}]%
\>[2]\AgdaPostulate{man}\AgdaSpace{}%
\AgdaSymbol{:}\AgdaSpace{}%
\AgdaFunction{CN}\<%
\\
%
\>[2]\AgdaPostulate{john}\AgdaSpace{}%
\AgdaSymbol{:}\AgdaSpace{}%
\AgdaPostulate{man}\<%
\end{code}

In Agda, there is only one sort of predicative universe, \term{Set}. In Coq
there are both impredicative and predicative universes, \term{Prop} and
\term{Set} respectively, of which \term{Type} is a superclass. While one defines
\term{CN := Set} in Coq, the type of impredicative propositions are included in
both \cite{fracoq} and \cite{luoCoq} which is not possible in Agda. It should be
possible to make everything predicative in Coq, but the authors' reasons for
using impredicativity were not explicated in their work. Agda's \term{Prop} are
by default proof irrelevant, whereas one must choose to make Coq's propositions
proof irrelevant. We don't explore more about the implications of foundations
here.

Once one has a the universe of common nouns, each of which may have many
inhabitants, we can ask how they are modified. Intransative Verbs (IVs) like
``walk", can be seen as a type restricted by the collection of things which have
the ability to walk, such as animals. We can see such verbs as functions taking a
specific type of common noun to an arbitrary type : $\llbracket IV \rrbracket\;
{:}\; (\llbracket x \rrbracket\; {:}\; \llbracket CN \rrbracket) \rightarrow
Set$

\begin{code}[hide]%
\>[0]\AgdaKeyword{postulate}\<%
\\
\>[0][@{}l@{\AgdaIndent{0}}]%
\>[2]\AgdaPostulate{animal}\AgdaSpace{}%
\AgdaSymbol{:}\AgdaSpace{}%
\AgdaPrimitive{Set}\<%
\end{code}
\begin{code}%
\>[0]\AgdaKeyword{postulate}\<%
\\
\>[0][@{}l@{\AgdaIndent{0}}]%
\>[2]\AgdaPostulate{walk}\AgdaSpace{}%
\AgdaSymbol{:}\AgdaSpace{}%
\AgdaPostulate{animal}\AgdaSpace{}%
\AgdaSymbol{->}\AgdaSpace{}%
\AgdaPrimitive{Set}\<%
\end{code}
\begin{code}[hide]%
\>[0]\AgdaKeyword{postulate}\<%
\\
\>[0][@{}l@{\AgdaIndent{0}}]%
\>[2]\AgdaPostulate{delegate}\AgdaSpace{}%
\AgdaPostulate{survey}\AgdaSpace{}%
\AgdaPostulate{object}\AgdaSpace{}%
\AgdaPostulate{surgeon}\AgdaSpace{}%
\AgdaPostulate{human}\AgdaSpace{}%
\AgdaSymbol{:}\AgdaSpace{}%
\AgdaFunction{CN}\<%
\end{code}

We can then encode the quantiers, noting that they also return just types the
dependent type \term{P} below is propositional in Coq. These are more arguably
more syntactically pleasing than our Mongtagovian semantics, because they only
bind a noun and a property about that noun, rather than rigidly requiring a verb
phrase and a noun phrase as arguments.

\begin{code}%
\>[0]\AgdaFunction{some}\AgdaSpace{}%
\AgdaSymbol{:}\AgdaSpace{}%
\AgdaSymbol{(}\AgdaBound{A}\AgdaSpace{}%
\AgdaSymbol{:}\AgdaSpace{}%
\AgdaFunction{CN}\AgdaSymbol{)}\AgdaSpace{}%
\AgdaSymbol{→}\AgdaSpace{}%
\AgdaSymbol{(}\AgdaBound{P}\AgdaSpace{}%
\AgdaSymbol{:}\AgdaSpace{}%
\AgdaBound{A}\AgdaSpace{}%
\AgdaSymbol{→}\AgdaSpace{}%
\AgdaPrimitive{Set}\AgdaSymbol{)}\AgdaSpace{}%
\AgdaSymbol{→}\AgdaSpace{}%
\AgdaPrimitive{Set}\<%
\\
\>[0]\AgdaFunction{some}\AgdaSpace{}%
\AgdaBound{A}\AgdaSpace{}%
\AgdaBound{P}\AgdaSpace{}%
\AgdaSymbol{=}\AgdaSpace{}%
\AgdaFunction{Σ[}\AgdaSpace{}%
\AgdaBound{x}\AgdaSpace{}%
\AgdaFunction{∈}\AgdaSpace{}%
\AgdaBound{A}\AgdaSpace{}%
\AgdaFunction{]}\AgdaSpace{}%
\AgdaBound{P}\AgdaSpace{}%
\AgdaBound{x}\<%
\\
%
\\[\AgdaEmptyExtraSkip]%
\>[0]\AgdaFunction{all}\AgdaSpace{}%
\AgdaSymbol{:}\AgdaSpace{}%
\AgdaSymbol{(}\AgdaBound{A}\AgdaSpace{}%
\AgdaSymbol{:}\AgdaSpace{}%
\AgdaFunction{CN}\AgdaSymbol{)}\AgdaSpace{}%
\AgdaSymbol{→}\AgdaSpace{}%
\AgdaSymbol{(}\AgdaBound{P}\AgdaSpace{}%
\AgdaSymbol{:}\AgdaSpace{}%
\AgdaBound{A}\AgdaSpace{}%
\AgdaSymbol{→}\AgdaSpace{}%
\AgdaPrimitive{Set}\AgdaSymbol{)}\AgdaSpace{}%
\AgdaSymbol{→}\AgdaSpace{}%
\AgdaPrimitive{Set}\<%
\\
\>[0]\AgdaFunction{all}\AgdaSpace{}%
\AgdaBound{A}\AgdaSpace{}%
\AgdaBound{P}\AgdaSpace{}%
\AgdaSymbol{=}\AgdaSpace{}%
\AgdaSymbol{(}\AgdaBound{x}\AgdaSpace{}%
\AgdaSymbol{:}\AgdaSpace{}%
\AgdaBound{A}\AgdaSymbol{)}\AgdaSpace{}%
\AgdaSymbol{→}\AgdaSpace{}%
\AgdaBound{P}\AgdaSpace{}%
\AgdaBound{x}\<%
\end{code}

Wanting to do inference with these examples, we hope to show that if John is a
man and every man walks, then John walks. The difficulty is that walk is a type
over animals, not men, and the relation between men and animals are not yet
covered by our type theory. The theory of coercive subtyping rectifies this and
gives a mechanism of implicity coercing the type of men to the type of animals,
as indeed all men are animals. One can form an order from the subtypes, with
possible infima and suprema, that may transform some abstract ontological model
of the domain into specific ways of using the information to prove inferences.

The coercions in coercive type theory can be approximated by the use of
Agda's instance arguements, which are indicated with \codeword{{{_}}} below.
Nonetheless, Agda doesn't support coercive subtyping as developed by Luo, and
therefore has weaknesses relative to Coq when it comes to eliminating ``coercion
bureaucracy". A coercion is a record type parameterized by two types $x$ and $y$
with one field \term{coe} which is merely a mapping from $x$ to $y$. We can then
compose and apply functions with arguements for which there exists an coercion.

\begin{code}%
\>[0]\AgdaKeyword{record}\AgdaSpace{}%
\AgdaRecord{Coercion}\AgdaSpace{}%
\AgdaSymbol{\{}\AgdaBound{a}\AgdaSymbol{\}}\AgdaSpace{}%
\AgdaSymbol{(}\AgdaBound{x}\AgdaSpace{}%
\AgdaBound{y}\AgdaSpace{}%
\AgdaSymbol{:}\AgdaSpace{}%
\AgdaPrimitive{Set}\AgdaSpace{}%
\AgdaBound{a}\AgdaSymbol{)}\AgdaSpace{}%
\AgdaSymbol{:}\AgdaSpace{}%
\AgdaPrimitive{Set}\AgdaSpace{}%
\AgdaBound{a}\AgdaSpace{}%
\AgdaKeyword{where}\<%
\\
\>[0][@{}l@{\AgdaIndent{0}}]%
\>[2]\AgdaKeyword{constructor}\AgdaSpace{}%
\AgdaOperator{\AgdaInductiveConstructor{⌞\AgdaUnderscore{}⌟}}\<%
\\
%
\>[2]\AgdaKeyword{field}\AgdaSpace{}%
\AgdaField{coe}\AgdaSpace{}%
\AgdaSymbol{:}\AgdaSpace{}%
\AgdaBound{x}\AgdaSpace{}%
\AgdaSymbol{→}\AgdaSpace{}%
\AgdaBound{y}\<%
\end{code}
\begin{code}[hide]%
\>[0]\AgdaKeyword{open}\AgdaSpace{}%
\AgdaModule{Coercion}\<%
\end{code}
\begin{code}%
\>[0]\AgdaOperator{\AgdaFunction{\AgdaUnderscore{}⊚\AgdaUnderscore{}}}\AgdaSpace{}%
\AgdaSymbol{:}\AgdaSpace{}%
\AgdaSymbol{∀}\AgdaSpace{}%
\AgdaSymbol{\{}\AgdaBound{a}\AgdaSymbol{\}}\AgdaSpace{}%
\AgdaSymbol{\{}\AgdaBound{A}\AgdaSpace{}%
\AgdaBound{B}\AgdaSpace{}%
\AgdaBound{C}\AgdaSpace{}%
\AgdaSymbol{:}\AgdaSpace{}%
\AgdaPrimitive{Set}\AgdaSpace{}%
\AgdaBound{a}\AgdaSymbol{\}}\AgdaSpace{}%
\AgdaSymbol{→}\AgdaSpace{}%
\AgdaRecord{Coercion}\AgdaSpace{}%
\AgdaBound{A}\AgdaSpace{}%
\AgdaBound{B}\AgdaSpace{}%
\AgdaSymbol{→}\AgdaSpace{}%
\AgdaRecord{Coercion}\AgdaSpace{}%
\AgdaBound{B}\AgdaSpace{}%
\AgdaBound{C}\AgdaSpace{}%
\AgdaSymbol{→}\AgdaSpace{}%
\AgdaRecord{Coercion}\AgdaSpace{}%
\AgdaBound{A}\AgdaSpace{}%
\AgdaBound{C}\<%
\\
\>[0]\AgdaOperator{\AgdaFunction{\AgdaUnderscore{}⊚\AgdaUnderscore{}}}\AgdaSpace{}%
\AgdaBound{c}\AgdaSpace{}%
\AgdaBound{d}\AgdaSpace{}%
\AgdaSymbol{=}\AgdaSpace{}%
\AgdaOperator{\AgdaInductiveConstructor{⌞}}\AgdaSpace{}%
\AgdaSymbol{(λ}\AgdaSpace{}%
\AgdaBound{x}\AgdaSpace{}%
\AgdaSymbol{→}\AgdaSpace{}%
\AgdaField{coe}\AgdaSpace{}%
\AgdaBound{d}\AgdaSpace{}%
\AgdaSymbol{(}\AgdaField{coe}\AgdaSpace{}%
\AgdaBound{c}\AgdaSpace{}%
\AgdaBound{x}\AgdaSymbol{))}\AgdaSpace{}%
\AgdaOperator{\AgdaInductiveConstructor{⌟}}\<%
\\
%
\\[\AgdaEmptyExtraSkip]%
\>[0]\AgdaOperator{\AgdaFunction{\AgdaUnderscore{}\$\AgdaUnderscore{}}}\AgdaSpace{}%
\AgdaSymbol{:}%
\>[154I]\AgdaSymbol{∀}\AgdaSpace{}%
\AgdaSymbol{\{}\AgdaBound{a}\AgdaSpace{}%
\AgdaBound{b}\AgdaSymbol{\}}\AgdaSpace{}%
\AgdaSymbol{\{}\AgdaBound{A}\AgdaSpace{}%
\AgdaBound{A′}\AgdaSpace{}%
\AgdaSymbol{:}\AgdaSpace{}%
\AgdaPrimitive{Set}\AgdaSpace{}%
\AgdaBound{a}\AgdaSymbol{\}}\AgdaSpace{}%
\AgdaSymbol{\{}\AgdaBound{B}\AgdaSpace{}%
\AgdaSymbol{:}\AgdaSpace{}%
\AgdaPrimitive{Set}\AgdaSpace{}%
\AgdaBound{b}\AgdaSymbol{\}}\AgdaSpace{}%
\AgdaSymbol{→}\AgdaSpace{}%
\AgdaSymbol{(}\AgdaBound{A}\AgdaSpace{}%
\AgdaSymbol{→}\AgdaSpace{}%
\AgdaBound{B}\AgdaSymbol{)}\AgdaSpace{}%
\AgdaSymbol{→}\<%
\\
\>[.][@{}l@{}]\<[154I]%
\>[6]\AgdaSymbol{\{\{}\AgdaBound{c}\AgdaSpace{}%
\AgdaSymbol{:}\AgdaSpace{}%
\AgdaRecord{Coercion}\AgdaSpace{}%
\AgdaBound{A′}\AgdaSpace{}%
\AgdaBound{A}\AgdaSpace{}%
\AgdaSymbol{\}\}}\AgdaSpace{}%
\AgdaSymbol{→}\AgdaSpace{}%
\AgdaBound{A′}\AgdaSpace{}%
\AgdaSymbol{→}\AgdaSpace{}%
\AgdaBound{B}\<%
\\
\>[0]\AgdaOperator{\AgdaFunction{\AgdaUnderscore{}\$\AgdaUnderscore{}}}\AgdaSpace{}%
\AgdaBound{f}\AgdaSpace{}%
\AgdaSymbol{\{\{}\AgdaBound{c}\AgdaSymbol{\}\}}\AgdaSpace{}%
\AgdaBound{a}\AgdaSpace{}%
\AgdaSymbol{=}\AgdaSpace{}%
\AgdaBound{f}\AgdaSpace{}%
\AgdaSymbol{(}\AgdaField{coe}\AgdaSpace{}%
\AgdaBound{c}\AgdaSpace{}%
\AgdaBound{a}\AgdaSymbol{)}\<%
\\
%
\\[\AgdaEmptyExtraSkip]%
\>[0]\AgdaKeyword{postulate}\<%
\\
\>[0][@{}l@{\AgdaIndent{0}}]%
\>[2]\AgdaPostulate{ha}\AgdaSpace{}%
\AgdaSymbol{:}\AgdaSpace{}%
\AgdaPostulate{human}\AgdaSpace{}%
\AgdaSymbol{→}\AgdaSpace{}%
\AgdaPostulate{animal}\<%
\\
%
\>[2]\AgdaPostulate{mh}\AgdaSpace{}%
\AgdaSymbol{:}\AgdaSpace{}%
\AgdaPostulate{man}\AgdaSpace{}%
\AgdaSymbol{→}\AgdaSpace{}%
\AgdaPostulate{human}\<%
\end{code}

The instance arguements, similar to Haskell's type-classes, allow one to
introduce the coercion information into a context so that one may compute with
these hidden typing relations.

\begin{code}%
\>[0]\AgdaKeyword{instance}\<%
\\
\>[0][@{}l@{\AgdaIndent{0}}]%
\>[2]\AgdaFunction{hac}\AgdaSpace{}%
\AgdaSymbol{=}\AgdaSpace{}%
\AgdaOperator{\AgdaInductiveConstructor{⌞}}\AgdaSpace{}%
\AgdaPostulate{ha}\AgdaSpace{}%
\AgdaOperator{\AgdaInductiveConstructor{⌟}}\<%
\\
%
\>[2]\AgdaFunction{mhc}\AgdaSpace{}%
\AgdaSymbol{=}\AgdaSpace{}%
\AgdaOperator{\AgdaInductiveConstructor{⌞}}\AgdaSpace{}%
\AgdaPostulate{mh}\AgdaSpace{}%
\AgdaOperator{\AgdaInductiveConstructor{⌟}}\<%
\\
%
\>[2]\AgdaFunction{mac}\AgdaSpace{}%
\AgdaSymbol{=}\AgdaSpace{}%
\AgdaFunction{mhc}\AgdaSpace{}%
\AgdaOperator{\AgdaFunction{⊚}}\AgdaSpace{}%
\AgdaFunction{hac}\<%
\end{code}

Once one has defined instances, Agda can infer that \term{walk} is a property of
men, which should be subtypes of animals. We must explictly explicitly declare
this in Agda, unfortunately. A type theory with native support for coercive
subtyping would save significant hassle, although someone with significant
experience using Agda's instance arguements might find a superior way to do this
rather than generating all the instances and coercion applications, possibly
without resorting to metaprogramming. However, once we have the infastracture in
place, we can not only infer basic facts about men, but also about animals and
their relation to men.

\begin{code}[hide]%
\>[0]\AgdaFunction{manwalk}\AgdaSpace{}%
\AgdaSymbol{:}\AgdaSpace{}%
\AgdaPostulate{man}\AgdaSpace{}%
\AgdaSymbol{→}\AgdaSpace{}%
\AgdaPrimitive{Set}\<%
\\
\>[0]\AgdaFunction{manwalk}\AgdaSpace{}%
\AgdaBound{m}\AgdaSpace{}%
\AgdaSymbol{=}\AgdaSpace{}%
\AgdaPostulate{walk}\AgdaSpace{}%
\AgdaOperator{\AgdaFunction{\$}}\AgdaSpace{}%
\AgdaBound{m}\<%
\end{code}
\begin{code}%
\>[0]\AgdaFunction{johnwalk}\AgdaSpace{}%
\AgdaSymbol{=}\AgdaSpace{}%
\AgdaFunction{manwalk}\AgdaSpace{}%
\AgdaPostulate{john}\<%
\\
\>[0]\AgdaFunction{allmanwalk}\AgdaSpace{}%
\AgdaSymbol{=}\AgdaSpace{}%
\AgdaFunction{all}\AgdaSpace{}%
\AgdaPostulate{man}\AgdaSpace{}%
\AgdaFunction{manwalk}\<%
\\
\>[0]\AgdaFunction{somemanwalk}\AgdaSpace{}%
\AgdaSymbol{=}\AgdaSpace{}%
\AgdaFunction{some}\AgdaSpace{}%
\AgdaPostulate{man}\AgdaSpace{}%
\AgdaFunction{manwalk}\<%
\\
%
\\[\AgdaEmptyExtraSkip]%
\>[0]\AgdaFunction{thm1}\AgdaSpace{}%
\AgdaSymbol{:}\AgdaSpace{}%
\AgdaFunction{allmanwalk}\AgdaSpace{}%
\AgdaSymbol{→}\AgdaSpace{}%
\AgdaFunction{johnwalk}\<%
\\
\>[0]\AgdaFunction{thm1}\AgdaSpace{}%
\AgdaBound{∀mWalk[m]}\AgdaSpace{}%
\AgdaSymbol{=}\AgdaSpace{}%
\AgdaBound{∀mWalk[m]}\AgdaSpace{}%
\AgdaPostulate{john}\<%
\\
%
\\[\AgdaEmptyExtraSkip]%
\>[0]\AgdaFunction{thm2}\AgdaSpace{}%
\AgdaSymbol{:}\AgdaSpace{}%
\AgdaFunction{johnwalk}\AgdaSpace{}%
\AgdaSymbol{→}\AgdaSpace{}%
\AgdaFunction{somemanwalk}\<%
\\
\>[0]\AgdaFunction{thm2}\AgdaSpace{}%
\AgdaBound{jw}\AgdaSpace{}%
\AgdaSymbol{=}\AgdaSpace{}%
\AgdaPostulate{john}\AgdaSpace{}%
\AgdaOperator{\AgdaInductiveConstructor{,}}\AgdaSpace{}%
\AgdaBound{jw}\<%
\\
%
\\[\AgdaEmptyExtraSkip]%
\>[0]\AgdaFunction{thm3}\AgdaSpace{}%
\AgdaSymbol{:}\AgdaSpace{}%
\AgdaFunction{somemanwalk}\AgdaSpace{}%
\AgdaSymbol{→}\AgdaSpace{}%
\AgdaFunction{some}\AgdaSpace{}%
\AgdaPostulate{animal}\AgdaSpace{}%
\AgdaPostulate{walk}\<%
\\
\>[0]\AgdaFunction{thm3}\AgdaSpace{}%
\AgdaSymbol{(}\AgdaBound{m}\AgdaSpace{}%
\AgdaOperator{\AgdaInductiveConstructor{,}}\AgdaSpace{}%
\AgdaBound{walk[m]}\AgdaSymbol{)}\AgdaSpace{}%
\AgdaSymbol{=}\AgdaSpace{}%
\AgdaPostulate{ha}\AgdaSpace{}%
\AgdaSymbol{(}\AgdaPostulate{mh}\AgdaSpace{}%
\AgdaBound{m}\AgdaSymbol{)}\AgdaSpace{}%
\AgdaOperator{\AgdaInductiveConstructor{,}}\AgdaSpace{}%
\AgdaBound{walk[m]}\<%
\end{code}

To the best of our knowledge, there is no way of coercing types directly, as in,
one cannot simply force the type-checker in \term{thm3} to accept the man
arguement without explicitly requiring the programmer to insert the coercions,
\term{ha (mh m)}. Another issue is that \term{manwalk} and \term{walk} are
explicitly different types, despite the instances allowing Agda to coerce the
fact that a man walks, \codeword{walk[m]}, to an animal walking. We may
reconcile this with more instance arguements, whereby we create a parameterized
record \term{Walks} with a single data point for the walking capacity. One can
then overload walks with all the different entities which can walk, and thereby
not have the ugly \term{manwalks} in the type signature of \term{thm3'}.

\begin{code}%
\>[0]\AgdaKeyword{record}\AgdaSpace{}%
\AgdaRecord{Walks}\AgdaSpace{}%
\AgdaSymbol{\{}\AgdaBound{a}\AgdaSymbol{\}}\AgdaSpace{}%
\AgdaSymbol{(}\AgdaBound{A}\AgdaSpace{}%
\AgdaSymbol{:}\AgdaSpace{}%
\AgdaPrimitive{Set}\AgdaSpace{}%
\AgdaBound{a}\AgdaSymbol{)}\AgdaSpace{}%
\AgdaSymbol{:}\AgdaSpace{}%
\AgdaPrimitive{Set}\AgdaSpace{}%
\AgdaBound{a}\AgdaSpace{}%
\AgdaKeyword{where}\<%
\\
\>[0][@{}l@{\AgdaIndent{0}}]%
\>[2]\AgdaKeyword{field}\<%
\\
\>[2][@{}l@{\AgdaIndent{0}}]%
\>[4]\AgdaField{walks}\AgdaSpace{}%
\AgdaSymbol{:}\AgdaSpace{}%
\AgdaBound{A}\AgdaSpace{}%
\AgdaSymbol{→}\AgdaSpace{}%
\AgdaPrimitive{Set}\<%
\\
%
\\[\AgdaEmptyExtraSkip]%
\>[0]\AgdaKeyword{open}\AgdaSpace{}%
\AgdaModule{Walks}\AgdaSpace{}%
\AgdaSymbol{\{\{...\}\}}\AgdaSpace{}%
\AgdaKeyword{public}\<%
\\
%
\\[\AgdaEmptyExtraSkip]%
\>[0]\AgdaKeyword{postulate}\<%
\\
\>[0][@{}l@{\AgdaIndent{0}}]%
\>[2]\AgdaPostulate{animalsWalk}\AgdaSpace{}%
\AgdaSymbol{:}\AgdaSpace{}%
\AgdaRecord{Walks}\AgdaSpace{}%
\AgdaPostulate{animal}\<%
\\
%
\\[\AgdaEmptyExtraSkip]%
\>[0]\AgdaKeyword{instance}\<%
\\
\>[0][@{}l@{\AgdaIndent{0}}]%
\>[2]\AgdaFunction{animalwalks}\AgdaSpace{}%
\AgdaSymbol{:}\AgdaSpace{}%
\AgdaRecord{Walks}\AgdaSpace{}%
\AgdaPostulate{animal}\<%
\\
%
\>[2]\AgdaFunction{animalwalks}\AgdaSpace{}%
\AgdaSymbol{=}\AgdaSpace{}%
\AgdaPostulate{animalsWalk}\<%
\\
%
\\[\AgdaEmptyExtraSkip]%
%
\>[2]\AgdaFunction{humanwalks}\AgdaSpace{}%
\AgdaSymbol{:}\AgdaSpace{}%
\AgdaRecord{Walks}\AgdaSpace{}%
\AgdaPostulate{human}\<%
\\
%
\>[2]\AgdaFunction{humanwalks}\AgdaSpace{}%
\AgdaSymbol{=}\AgdaSpace{}%
\AgdaKeyword{record}\AgdaSpace{}%
\AgdaSymbol{\{}\AgdaSpace{}%
\AgdaField{walks}\AgdaSpace{}%
\AgdaSymbol{=}\AgdaSpace{}%
\AgdaSymbol{λ}\AgdaSpace{}%
\AgdaBound{h}\AgdaSpace{}%
\AgdaSymbol{→}\AgdaSpace{}%
\AgdaField{Walks.walks}\AgdaSpace{}%
\AgdaPostulate{animalsWalk}\AgdaSpace{}%
\AgdaOperator{\AgdaFunction{\$}}\AgdaSpace{}%
\AgdaBound{h}\AgdaSymbol{\}}\<%
\\
%
\\[\AgdaEmptyExtraSkip]%
%
\>[2]\AgdaFunction{manwalks}\AgdaSpace{}%
\AgdaSymbol{:}\AgdaSpace{}%
\AgdaRecord{Walks}\AgdaSpace{}%
\AgdaPostulate{man}\<%
\\
%
\>[2]\AgdaFunction{manwalks}\AgdaSpace{}%
\AgdaSymbol{=}\AgdaSpace{}%
\AgdaKeyword{record}\AgdaSpace{}%
\AgdaSymbol{\{}\AgdaSpace{}%
\AgdaField{walks}\AgdaSpace{}%
\AgdaSymbol{=}\AgdaSpace{}%
\AgdaSymbol{λ}\AgdaSpace{}%
\AgdaBound{m}\AgdaSpace{}%
\AgdaSymbol{→}\AgdaSpace{}%
\AgdaField{Walks.walks}\AgdaSpace{}%
\AgdaPostulate{animalsWalk}\AgdaSpace{}%
\AgdaOperator{\AgdaFunction{\$}}\AgdaSpace{}%
\AgdaBound{m}\AgdaSymbol{\}}\<%
\\
%
\\[\AgdaEmptyExtraSkip]%
\>[0]\AgdaFunction{thm3'}\AgdaSpace{}%
\AgdaSymbol{:}\AgdaSpace{}%
\AgdaFunction{some}\AgdaSpace{}%
\AgdaPostulate{man}\AgdaSpace{}%
\AgdaField{walks}%
\>[24]\AgdaSymbol{→}\AgdaSpace{}%
\AgdaFunction{some}\AgdaSpace{}%
\AgdaPostulate{human}\AgdaSpace{}%
\AgdaField{walks}\<%
\\
\>[0]\AgdaFunction{thm3'}\AgdaSpace{}%
\AgdaSymbol{(}\AgdaBound{m}\AgdaSpace{}%
\AgdaOperator{\AgdaInductiveConstructor{,}}\AgdaSpace{}%
\AgdaBound{walk[m]}\AgdaSymbol{)}\AgdaSpace{}%
\AgdaSymbol{=}\AgdaSpace{}%
\AgdaPostulate{mh}\AgdaSpace{}%
\AgdaBound{m}\AgdaSpace{}%
\AgdaOperator{\AgdaInductiveConstructor{,}}\AgdaSpace{}%
\AgdaBound{walk[m]}\<%
\end{code}

\subsection{Irish Delegate Example}

We finish with the following FraCas example, which includes the ditransative
verb ``finished", the adjective ``Irish", and adverb ``on time", and the
determiner ``the". We include a common noun for \term{object}, of which
\term{survey} and \term{animal} should be subtypes.

\begin{verbatim}
Premise  : Some Irish delegates finished the survey on time.
Question : Did any delegates finish the survey on time?
Answer   : Yes.
\end{verbatim}

\begin{code}[hide]%
\>[0]\AgdaKeyword{postulate}\<%
\\
\>[0][@{}l@{\AgdaIndent{0}}]%
\>[2]\AgdaPostulate{ao}\AgdaSpace{}%
\AgdaSymbol{:}\AgdaSpace{}%
\AgdaPostulate{animal}\AgdaSpace{}%
\AgdaSymbol{→}\AgdaSpace{}%
\AgdaPostulate{object}\<%
\\
%
\>[2]\AgdaPostulate{dh}\AgdaSpace{}%
\AgdaSymbol{:}\AgdaSpace{}%
\AgdaPostulate{delegate}\AgdaSpace{}%
\AgdaSymbol{→}\AgdaSpace{}%
\AgdaPostulate{human}\<%
\\
%
\>[2]\AgdaPostulate{so}\AgdaSpace{}%
\AgdaSymbol{:}\AgdaSpace{}%
\AgdaPostulate{survey}\AgdaSpace{}%
\AgdaSymbol{→}\AgdaSpace{}%
\AgdaPostulate{object}\<%
\\
\>[0]\AgdaKeyword{instance}\<%
\\
\>[0][@{}l@{\AgdaIndent{0}}]%
\>[2]\AgdaFunction{aoc}\AgdaSpace{}%
\AgdaSymbol{=}\AgdaSpace{}%
\AgdaOperator{\AgdaInductiveConstructor{⌞}}\AgdaSpace{}%
\AgdaPostulate{ao}\AgdaSpace{}%
\AgdaOperator{\AgdaInductiveConstructor{⌟}}\<%
\\
%
\>[2]\AgdaFunction{dhc}\AgdaSpace{}%
\AgdaSymbol{=}\AgdaSpace{}%
\AgdaOperator{\AgdaInductiveConstructor{⌞}}\AgdaSpace{}%
\AgdaPostulate{dh}\AgdaSpace{}%
\AgdaOperator{\AgdaInductiveConstructor{⌟}}\<%
\\
%
\>[2]\AgdaFunction{soc}\AgdaSpace{}%
\AgdaSymbol{=}\AgdaSpace{}%
\AgdaOperator{\AgdaInductiveConstructor{⌞}}\AgdaSpace{}%
\AgdaPostulate{so}\AgdaSpace{}%
\AgdaOperator{\AgdaInductiveConstructor{⌟}}\<%
\\
%
\>[2]\AgdaFunction{dac}\AgdaSpace{}%
\AgdaSymbol{=}\AgdaSpace{}%
\AgdaFunction{dhc}\AgdaSpace{}%
\AgdaOperator{\AgdaFunction{⊚}}\AgdaSpace{}%
\AgdaFunction{hac}\AgdaSpace{}%
\AgdaComment{--added}\<%
\\
%
\>[2]\AgdaFunction{hoc}\AgdaSpace{}%
\AgdaSymbol{=}\AgdaSpace{}%
\AgdaFunction{hac}\AgdaSpace{}%
\AgdaOperator{\AgdaFunction{⊚}}\AgdaSpace{}%
\AgdaFunction{aoc}\<%
\\
%
\>[2]\AgdaFunction{doc}\AgdaSpace{}%
\AgdaSymbol{=}\AgdaSpace{}%
\AgdaSymbol{(}\AgdaFunction{dhc}\AgdaSpace{}%
\AgdaOperator{\AgdaFunction{⊚}}\AgdaSpace{}%
\AgdaFunction{hac}\AgdaSymbol{)}\AgdaSpace{}%
\AgdaOperator{\AgdaFunction{⊚}}\AgdaSpace{}%
\AgdaFunction{aoc}\<%
\end{code}

Semantically, Ditransitive Verbs (DVs) are similair to IVs, they are just binary
functions instead of unary.

$$\llbracket DV \rrbracket\; {:}\; (\llbracket x \rrbracket\; {:}\; \llbracket
CN \rrbracket) \rightarrow (\llbracket y \rrbracket\; {:}\; \llbracket CN
\rrbracket) \rightarrow Set$$

The quality of being on time, which modifies a verb, is intpreted as a function
which takes a common noun $cn$, a type indexed by $cn$ (the verb), and returns a
type which is itself dependent on $cn$. The intuition that one can continue to
modify a verb phrase with more adverbs is immediately obvious based of the type
signature, because it returns the same type as a verb after taking a verb as an
arguement.

$$\llbracket ADV \rrbracket\; {:}\; (\Pi \; x \; {:}\;
\llbracket CN \rrbracket) \rightarrow (\llbracket x \rrbracket\; \rightarrow
Set) \rightarrow (\llbracket x \rrbracket\; \rightarrow Set)$$

The determiner ``the" is simply a way of extracting a member from a given CN.

$$\llbracket the \rrbracket\; {:}\; (\Pi \; x \; {:}\; \llbracket CN \rrbracket) \rightarrow x)$$

Finally, the MTT interpretation of adjectives is definitionally equal to IVs,
$\llbracket ADJ \rrbracket\; {:}\; (\llbracket x \rrbracket\; {:}\; \llbracket
CN \rrbracket) \rightarrow Set$. This does not mean they are semantically at all
similar. Verbs describe what an individual does, whereas adjectives describe
some property of the individual. To apply an adjective $a$ to a member $n$ of
some CN gives a sentence whose meaning is ``$a$ is $n$", whereby the syntactic
``is" is implicit in the semantics.

\begin{code}%
\>[0]\AgdaKeyword{postulate}\<%
\\
\>[0][@{}l@{\AgdaIndent{0}}]%
\>[2]\AgdaPostulate{finish}\AgdaSpace{}%
\AgdaSymbol{:}\AgdaSpace{}%
\AgdaPostulate{object}\AgdaSpace{}%
\AgdaSymbol{→}\AgdaSpace{}%
\AgdaPostulate{human}\AgdaSpace{}%
\AgdaSymbol{→}\AgdaSpace{}%
\AgdaPrimitive{Set}\<%
\\
%
\>[2]\AgdaPostulate{ontime}\AgdaSpace{}%
\AgdaSymbol{:}\AgdaSpace{}%
\AgdaSymbol{(}\AgdaBound{A}\AgdaSpace{}%
\AgdaSymbol{:}\AgdaSpace{}%
\AgdaFunction{CN}\AgdaSymbol{)}\AgdaSpace{}%
\AgdaSymbol{→}\AgdaSpace{}%
\AgdaSymbol{(}\AgdaBound{A}\AgdaSpace{}%
\AgdaSymbol{→}\AgdaSpace{}%
\AgdaPrimitive{Set}\AgdaSymbol{)}\AgdaSpace{}%
\AgdaSymbol{→}\AgdaSpace{}%
\AgdaSymbol{(}\AgdaBound{A}\AgdaSpace{}%
\AgdaSymbol{→}\AgdaSpace{}%
\AgdaPrimitive{Set}\AgdaSymbol{)}\<%
\\
%
\>[2]\AgdaPostulate{the}\AgdaSpace{}%
\AgdaSymbol{:}\AgdaSpace{}%
\AgdaSymbol{(}\AgdaBound{A}\AgdaSpace{}%
\AgdaSymbol{:}\AgdaSpace{}%
\AgdaFunction{CN}\AgdaSymbol{)}\AgdaSpace{}%
\AgdaSymbol{→}%
\>[20]\AgdaBound{A}\<%
\\
%
\>[2]\AgdaPostulate{irish}\AgdaSpace{}%
\AgdaSymbol{:}\AgdaSpace{}%
\AgdaPostulate{object}\AgdaSpace{}%
\AgdaSymbol{→}\AgdaSpace{}%
\AgdaPrimitive{Set}\<%
\end{code}

Adjectives are generally not meant to return sentences, but other common
nouns. Therefore, we can leverage the dependent product type or records more
generally to describe modified common nouns, whereby the first element $c$ is a
member of some CN and the second member is a proof that $c$ has the property the
adjective expresses. We can therefore see the example of \term{irishdelegate} as
such in Agda:

\begin{code}[hide]%
\>[0]\AgdaFunction{dobj}\AgdaSpace{}%
\AgdaSymbol{:}\AgdaSpace{}%
\AgdaPostulate{delegate}\AgdaSpace{}%
\AgdaSymbol{→}\AgdaSpace{}%
\AgdaPostulate{object}\<%
\\
\>[0]\AgdaFunction{dobj}\AgdaSpace{}%
\AgdaBound{del}\AgdaSpace{}%
\AgdaSymbol{=}\AgdaSpace{}%
\AgdaPostulate{ao}\AgdaSpace{}%
\AgdaSymbol{(}\AgdaPostulate{ha}\AgdaSpace{}%
\AgdaSymbol{(}\AgdaPostulate{dh}\AgdaSpace{}%
\AgdaBound{del}\AgdaSymbol{))}\<%
\end{code}
\begin{code}%
\>[0]\AgdaKeyword{record}\AgdaSpace{}%
\AgdaRecord{irishdelegate}\AgdaSpace{}%
\AgdaSymbol{:}\AgdaSpace{}%
\AgdaFunction{CN}\AgdaSpace{}%
\AgdaKeyword{where}\<%
\\
\>[0][@{}l@{\AgdaIndent{0}}]%
\>[2]\AgdaKeyword{constructor}\<%
\\
\>[2][@{}l@{\AgdaIndent{0}}]%
\>[4]\AgdaInductiveConstructor{mkIrishdelegate}\<%
\\
%
\>[2]\AgdaKeyword{field}\<%
\\
\>[2][@{}l@{\AgdaIndent{0}}]%
\>[4]\AgdaField{c}\AgdaSpace{}%
\AgdaSymbol{:}\AgdaSpace{}%
\AgdaPostulate{delegate}\<%
\\
%
\>[4]\AgdaField{ic}\AgdaSpace{}%
\AgdaSymbol{:}\AgdaSpace{}%
\AgdaPostulate{irish}\AgdaSpace{}%
\AgdaOperator{\AgdaFunction{\$}}\AgdaSpace{}%
\AgdaField{c}\<%
\end{code}

We can follow the same methodology as before, coercing Irish delegates to
delegates axiomatically, and then applying the semantic interpretations of the
words such that the types align correctly - where one sees this actually follows
from an intuitive syntactic presentation.

\begin{code}[hide]%
\>[0]\AgdaKeyword{postulate}\<%
\\
\>[0][@{}l@{\AgdaIndent{0}}]%
\>[2]\AgdaPostulate{idd}\AgdaSpace{}%
\AgdaSymbol{:}\AgdaSpace{}%
\AgdaRecord{irishdelegate}\AgdaSpace{}%
\AgdaSymbol{→}\AgdaSpace{}%
\AgdaPostulate{delegate}\<%
\\
%
\\[\AgdaEmptyExtraSkip]%
\>[0]\AgdaKeyword{instance}\<%
\\
\>[0][@{}l@{\AgdaIndent{0}}]%
\>[2]\AgdaFunction{iddc}\AgdaSpace{}%
\AgdaSymbol{=}\AgdaSpace{}%
\AgdaOperator{\AgdaInductiveConstructor{⌞}}\AgdaSpace{}%
\AgdaPostulate{idd}\AgdaSpace{}%
\AgdaOperator{\AgdaInductiveConstructor{⌟}}\<%
\\
%
\>[2]\AgdaFunction{idh}\AgdaSpace{}%
\AgdaSymbol{=}\AgdaSpace{}%
\AgdaFunction{iddc}\AgdaSpace{}%
\AgdaOperator{\AgdaFunction{⊚}}\AgdaSpace{}%
\AgdaFunction{dhc}\<%
\end{code}
\begin{code}[hide]%
\>[0]\AgdaFunction{finishTheSurvey}\AgdaSpace{}%
\AgdaSymbol{:}\AgdaSpace{}%
\AgdaPostulate{human}\AgdaSpace{}%
\AgdaSymbol{→}\AgdaSpace{}%
\AgdaPrimitive{Set}\<%
\\
\>[0]\AgdaFunction{finishTheSurvey}\AgdaSpace{}%
\AgdaSymbol{=}\AgdaSpace{}%
\AgdaPostulate{finish}\AgdaSpace{}%
\AgdaOperator{\AgdaFunction{\$}}\AgdaSpace{}%
\AgdaSymbol{(}\AgdaPostulate{the}\AgdaSpace{}%
\AgdaPostulate{survey}\AgdaSymbol{)}\<%
\\
%
\\[\AgdaEmptyExtraSkip]%
\>[0]\AgdaFunction{finishedTheSurveyOnTime}\AgdaSpace{}%
\AgdaSymbol{:}\AgdaSpace{}%
\AgdaPostulate{delegate}\AgdaSpace{}%
\AgdaSymbol{→}\AgdaSpace{}%
\AgdaPrimitive{Set}\<%
\\
\>[0]\AgdaFunction{finishedTheSurveyOnTime}\AgdaSpace{}%
\AgdaBound{x}\AgdaSpace{}%
\AgdaSymbol{=}\AgdaSpace{}%
\AgdaPostulate{ontime}\AgdaSpace{}%
\AgdaPostulate{human}\AgdaSpace{}%
\AgdaFunction{finishTheSurvey}\AgdaSpace{}%
\AgdaOperator{\AgdaFunction{\$}}\AgdaSpace{}%
\AgdaBound{x}\<%
\\
%
\\[\AgdaEmptyExtraSkip]%
\>[0]\AgdaFunction{someDelegateFinishedTheSurveyOnTime}\AgdaSpace{}%
\AgdaSymbol{:}\AgdaSpace{}%
\AgdaPrimitive{Set}\<%
\\
\>[0]\AgdaFunction{someDelegateFinishedTheSurveyOnTime}\AgdaSpace{}%
\AgdaSymbol{=}\AgdaSpace{}%
\AgdaFunction{some}\AgdaSpace{}%
\AgdaPostulate{delegate}\AgdaSpace{}%
\AgdaFunction{finishedTheSurveyOnTime}\<%
\end{code}

Once one builds a parallel infastructure for \term{irishdelegate}, one can then
proceed with the inference. We note that the work has to be doubled because
\term{finishedTheSurveyOnTime} and \term{someDelegateFinishedTheSurveyOnTime}
need to be refactored, renaming \term{delegate} to \term{irishdelegate}. Again,
this inference is just the identity function modulo an explicit \term{idd}
coercion, and implicit coercions allowing \codeword{finishedOnTime} to be cast to
its most general formulation where it is parameterized \term{human}.

\begin{code}[hide]%
\>[0]\AgdaFunction{finishedTheSurveyOnTime'}\AgdaSpace{}%
\AgdaSymbol{:}\AgdaSpace{}%
\AgdaRecord{irishdelegate}\AgdaSpace{}%
\AgdaSymbol{→}\AgdaSpace{}%
\AgdaPrimitive{Set}\<%
\\
\>[0]\AgdaFunction{finishedTheSurveyOnTime'}\AgdaSpace{}%
\AgdaBound{x}\AgdaSpace{}%
\AgdaSymbol{=}\AgdaSpace{}%
\AgdaPostulate{ontime}\AgdaSpace{}%
\AgdaPostulate{human}\AgdaSpace{}%
\AgdaFunction{finishTheSurvey}\AgdaSpace{}%
\AgdaOperator{\AgdaFunction{\$}}\AgdaSpace{}%
\AgdaBound{x}\<%
\\
%
\\[\AgdaEmptyExtraSkip]%
\>[0]\AgdaFunction{someIrishDelegateFinishedTheSurveyOnTime}\AgdaSpace{}%
\AgdaSymbol{:}\AgdaSpace{}%
\AgdaPrimitive{Set}\<%
\\
\>[0]\AgdaFunction{someIrishDelegateFinishedTheSurveyOnTime}\AgdaSpace{}%
\AgdaSymbol{=}\AgdaSpace{}%
\AgdaFunction{some}\AgdaSpace{}%
\AgdaRecord{irishdelegate}\AgdaSpace{}%
\AgdaFunction{finishedTheSurveyOnTime'}\<%
\end{code}
\begin{code}%
\>[0]\AgdaFunction{fc55}\AgdaSpace{}%
\AgdaSymbol{:}\<%
\\
\>[0][@{}l@{\AgdaIndent{0}}]%
\>[2]\AgdaFunction{someIrishDelegateFinishedTheSurveyOnTime}\AgdaSpace{}%
\AgdaSymbol{→}\AgdaSpace{}%
\AgdaFunction{someDelegateFinishedTheSurveyOnTime}\<%
\\
\>[0]\AgdaFunction{fc55}\AgdaSpace{}%
\AgdaSymbol{(}\AgdaBound{irishDelegate}\AgdaSpace{}%
\AgdaOperator{\AgdaInductiveConstructor{,}}\AgdaSpace{}%
\AgdaBound{finishedOnTime}\AgdaSymbol{)}\AgdaSpace{}%
\AgdaSymbol{=}\AgdaSpace{}%
\AgdaSymbol{(}\AgdaPostulate{idd}\AgdaSpace{}%
\AgdaBound{irishDelegate}\AgdaSymbol{)}\AgdaSpace{}%
\AgdaOperator{\AgdaInductiveConstructor{,}}\AgdaSpace{}%
\AgdaBound{finishedOnTime}\<%
\end{code}

We note that one could have instead included an extensionality clause for
adjectives and adverbs, wherby one gives additional information so that the
arguement and return types, dependent on some CN $A$, behave coherently with
respect to arbitrary arguements of $A$. One can then derive the adverb by
forgetting the extensionality clause. The inference works out the same.

\begin{code}%
\>[0]\AgdaKeyword{postulate}\<%
\\
\>[0][@{}l@{\AgdaIndent{0}}]%
\>[2]\AgdaPostulate{ADV}\AgdaSpace{}%
\AgdaSymbol{:}\AgdaSpace{}%
\AgdaSymbol{(}\AgdaBound{A}\AgdaSpace{}%
\AgdaSymbol{:}\AgdaSpace{}%
\AgdaFunction{CN}\AgdaSymbol{)}\AgdaSpace{}%
\AgdaSymbol{(}\AgdaBound{v}\AgdaSpace{}%
\AgdaSymbol{:}\AgdaSpace{}%
\AgdaBound{A}\AgdaSpace{}%
\AgdaSymbol{→}\AgdaSpace{}%
\AgdaPrimitive{Set}\AgdaSymbol{)}\AgdaSpace{}%
\AgdaSymbol{→}\AgdaSpace{}%
\AgdaFunction{Σ[}\AgdaSpace{}%
\AgdaBound{p}\AgdaSpace{}%
\AgdaFunction{∈}\AgdaSpace{}%
\AgdaSymbol{(}\AgdaBound{A}\AgdaSpace{}%
\AgdaSymbol{→}\AgdaSpace{}%
\AgdaPrimitive{Set}\AgdaSymbol{)}\AgdaSpace{}%
\AgdaFunction{]}\AgdaSpace{}%
\AgdaSymbol{((}\AgdaBound{x}\AgdaSpace{}%
\AgdaSymbol{:}\AgdaSpace{}%
\AgdaBound{A}\AgdaSymbol{)}\AgdaSpace{}%
\AgdaSymbol{→}\AgdaSpace{}%
\AgdaBound{p}\AgdaSpace{}%
\AgdaBound{x}\AgdaSpace{}%
\AgdaSymbol{→}\AgdaSpace{}%
\AgdaBound{v}\AgdaSpace{}%
\AgdaBound{x}\AgdaSymbol{)}\<%
\\
%
\\[\AgdaEmptyExtraSkip]%
\>[0]\AgdaOperator{\AgdaFunction{on\AgdaUnderscore{}time}}\AgdaSpace{}%
\AgdaSymbol{:}\AgdaSpace{}%
\AgdaSymbol{(}\AgdaBound{A}\AgdaSpace{}%
\AgdaSymbol{:}\AgdaSpace{}%
\AgdaFunction{CN}\AgdaSymbol{)}\AgdaSpace{}%
\AgdaSymbol{(}\AgdaBound{v}\AgdaSpace{}%
\AgdaSymbol{:}\AgdaSpace{}%
\AgdaBound{A}\AgdaSpace{}%
\AgdaSymbol{→}\AgdaSpace{}%
\AgdaPrimitive{Set}\AgdaSymbol{)}\AgdaSpace{}%
\AgdaSymbol{→}\AgdaSpace{}%
\AgdaBound{A}\AgdaSpace{}%
\AgdaSymbol{→}\AgdaSpace{}%
\AgdaPrimitive{Set}\<%
\\
\>[0]\AgdaOperator{\AgdaFunction{on\AgdaUnderscore{}time}}\AgdaSpace{}%
\AgdaBound{A}\AgdaSpace{}%
\AgdaBound{v}\AgdaSpace{}%
\AgdaSymbol{=}\AgdaSpace{}%
\AgdaField{proj₁}\AgdaSpace{}%
\AgdaSymbol{(}\AgdaPostulate{ADV}\AgdaSpace{}%
\AgdaBound{A}\AgdaSpace{}%
\AgdaBound{v}\AgdaSymbol{)}\<%
\end{code}
\begin{code}[hide]%
\>[0]\AgdaFunction{2finishedTheSurveyOnTime}\AgdaSpace{}%
\AgdaSymbol{:}\AgdaSpace{}%
\AgdaPostulate{delegate}\AgdaSpace{}%
\AgdaSymbol{→}\AgdaSpace{}%
\AgdaPrimitive{Set}\<%
\\
\>[0]\AgdaFunction{2finishedTheSurveyOnTime}\AgdaSpace{}%
\AgdaBound{x}\AgdaSpace{}%
\AgdaSymbol{=}\AgdaSpace{}%
\AgdaOperator{\AgdaFunction{on\AgdaUnderscore{}time}}\AgdaSpace{}%
\AgdaPostulate{human}\AgdaSpace{}%
\AgdaFunction{finishTheSurvey}\AgdaSpace{}%
\AgdaOperator{\AgdaFunction{\$}}\AgdaSpace{}%
\AgdaBound{x}\<%
\\
%
\\[\AgdaEmptyExtraSkip]%
\>[0]\AgdaFunction{2finishedTheSurveyOnTime'}\AgdaSpace{}%
\AgdaSymbol{:}\AgdaSpace{}%
\AgdaRecord{irishdelegate}\AgdaSpace{}%
\AgdaSymbol{→}\AgdaSpace{}%
\AgdaPrimitive{Set}\<%
\\
\>[0]\AgdaFunction{2finishedTheSurveyOnTime'}\AgdaSpace{}%
\AgdaBound{x}\AgdaSpace{}%
\AgdaSymbol{=}\AgdaSpace{}%
\AgdaOperator{\AgdaFunction{on\AgdaUnderscore{}time}}\AgdaSpace{}%
\AgdaPostulate{human}\AgdaSpace{}%
\AgdaFunction{finishTheSurvey}\AgdaSpace{}%
\AgdaOperator{\AgdaFunction{\$}}\AgdaSpace{}%
\AgdaBound{x}\<%
\\
%
\\[\AgdaEmptyExtraSkip]%
\>[0]\AgdaFunction{2someIrishDelegateFinishedTheSurveyOnTime}\AgdaSpace{}%
\AgdaSymbol{:}\AgdaSpace{}%
\AgdaPrimitive{Set}\<%
\\
\>[0]\AgdaFunction{2someIrishDelegateFinishedTheSurveyOnTime}\AgdaSpace{}%
\AgdaSymbol{=}\AgdaSpace{}%
\AgdaFunction{some}\AgdaSpace{}%
\AgdaRecord{irishdelegate}\AgdaSpace{}%
\AgdaFunction{2finishedTheSurveyOnTime'}\<%
\\
%
\\[\AgdaEmptyExtraSkip]%
\>[0]\AgdaFunction{2someDelegateFinishedTheSurveyOnTime}\AgdaSpace{}%
\AgdaSymbol{:}\AgdaSpace{}%
\AgdaPrimitive{Set}\<%
\\
\>[0]\AgdaFunction{2someDelegateFinishedTheSurveyOnTime}\AgdaSpace{}%
\AgdaSymbol{=}\AgdaSpace{}%
\AgdaFunction{some}\AgdaSpace{}%
\AgdaPostulate{delegate}\AgdaSpace{}%
\AgdaFunction{2finishedTheSurveyOnTime}\<%
\end{code}
\begin{code}%
\>[0]\AgdaFunction{fc55'}\AgdaSpace{}%
\AgdaSymbol{:}\<%
\\
\>[0][@{}l@{\AgdaIndent{0}}]%
\>[2]\AgdaFunction{2someIrishDelegateFinishedTheSurveyOnTime}\AgdaSpace{}%
\AgdaSymbol{→}\AgdaSpace{}%
\AgdaFunction{2someDelegateFinishedTheSurveyOnTime}\<%
\\
\>[0]\AgdaFunction{fc55'}\AgdaSpace{}%
\AgdaSymbol{(}\AgdaBound{irishDelegate}\AgdaSpace{}%
\AgdaOperator{\AgdaInductiveConstructor{,}}\AgdaSpace{}%
\AgdaBound{finishedOnTime}\AgdaSymbol{)}\AgdaSpace{}%
\AgdaSymbol{=}\AgdaSpace{}%
\AgdaSymbol{(}\AgdaPostulate{idd}\AgdaSpace{}%
\AgdaBound{irishDelegate}\AgdaSymbol{)}\AgdaSpace{}%
\AgdaOperator{\AgdaInductiveConstructor{,}}\AgdaSpace{}%
\AgdaBound{finishedOnTime}\<%
\end{code}

We now investigate the possiblity of gereralizing Irish, as well as integrating
the adjectival work with our previous work generating instance arguements for
``walks".

Unlike walking, which was assumed to apply to all animals, being Irish is a
restriction on the set of objects of some given domain. Therefore we can't just
define the record parametrically for all common nouns, but rather must include
an instance arguement for the coercion. Note this would break the semantic model
if we were to include the type of common noun ``Swede" with a coercion to
humans, because one would be able to make an Irish Swede.

\begin{code}%
\>[0]\AgdaKeyword{record}\AgdaSpace{}%
\AgdaRecord{irishThing}\AgdaSpace{}%
\AgdaSymbol{(}\AgdaBound{A}\AgdaSpace{}%
\AgdaSymbol{:}\AgdaSpace{}%
\AgdaFunction{CN}\AgdaSymbol{)}\AgdaSpace{}%
\AgdaSymbol{\{\{}\AgdaBound{c}\AgdaSpace{}%
\AgdaSymbol{:}\AgdaSpace{}%
\AgdaRecord{Coercion}\AgdaSpace{}%
\AgdaBound{A}\AgdaSpace{}%
\AgdaPostulate{object}\AgdaSymbol{\}\}}\AgdaSpace{}%
\AgdaSymbol{:}\AgdaSpace{}%
\AgdaFunction{CN}\AgdaSpace{}%
\AgdaKeyword{where}\<%
\\
\>[0][@{}l@{\AgdaIndent{0}}]%
\>[2]\AgdaKeyword{constructor}\<%
\\
\>[2][@{}l@{\AgdaIndent{0}}]%
\>[4]\AgdaInductiveConstructor{mkIrish}\<%
\\
%
\>[2]\AgdaKeyword{field}\<%
\\
\>[2][@{}l@{\AgdaIndent{0}}]%
\>[4]\AgdaField{thing}\AgdaSpace{}%
\AgdaSymbol{:}\AgdaSpace{}%
\AgdaBound{A}\<%
\\
%
\>[4]\AgdaField{isIrish}\AgdaSpace{}%
\AgdaSymbol{:}\AgdaSpace{}%
\AgdaPostulate{irish}\AgdaSpace{}%
\AgdaOperator{\AgdaFunction{\$}}\AgdaSpace{}%
\AgdaField{thing}\<%
\end{code}
\begin{code}[hide]%
\>[0]\AgdaKeyword{open}\AgdaSpace{}%
\AgdaModule{irishThing}\AgdaSpace{}%
\AgdaSymbol{\{\{...\}\}}\AgdaSpace{}%
\AgdaKeyword{public}\<%
\end{code}

Once can now delcare Irish entities using the record for humans, delegates, and
animals, where one can include the coercion arguements explicitly, even though
they are inferrable. Thereafter, we can overload walks even more. Although it is
clear that a lot of this code is boilerplate, the instance declarations must be
nullary, and basic code generation techniques would be needed to scale this to a
larger corpus. The point is, once we know that animals walk, anything subsumed
under that category is straightforward to make ``walkable".

\begin{code}%
\>[0]\AgdaFunction{IrishDelegate}\AgdaSpace{}%
\AgdaSymbol{=}\AgdaSpace{}%
\AgdaRecord{irishThing}\AgdaSpace{}%
\AgdaPostulate{delegate}\AgdaSpace{}%
\AgdaSymbol{\{\{}\AgdaFunction{doc}\AgdaSymbol{\}\}}\<%
\\
\>[0]\AgdaFunction{IrishHuman}\AgdaSpace{}%
\AgdaSymbol{=}\AgdaSpace{}%
\AgdaRecord{irishThing}\AgdaSpace{}%
\AgdaPostulate{human}\AgdaSpace{}%
\AgdaSymbol{\{\{}\AgdaFunction{hoc}\AgdaSymbol{\}\}}\<%
\\
\>[0]\AgdaFunction{IrishAnimal}\AgdaSpace{}%
\AgdaSymbol{=}\AgdaSpace{}%
\AgdaRecord{irishThing}\AgdaSpace{}%
\AgdaPostulate{animal}\AgdaSpace{}%
\AgdaSymbol{\{\{}\AgdaFunction{aoc}\AgdaSymbol{\}\}}\<%
\\
%
\\[\AgdaEmptyExtraSkip]%
\>[0]\AgdaKeyword{instance}\<%
\\
\>[0][@{}l@{\AgdaIndent{0}}]%
\>[2]\AgdaFunction{irishAnimalWalks}\AgdaSpace{}%
\AgdaSymbol{:}\AgdaSpace{}%
\AgdaRecord{Walks}\AgdaSpace{}%
\AgdaFunction{IrishAnimal}\<%
\\
%
\>[2]\AgdaFunction{irishAnimalWalks}\AgdaSpace{}%
\AgdaSymbol{=}\AgdaSpace{}%
\AgdaKeyword{record}\AgdaSpace{}%
\AgdaSymbol{\{}\AgdaSpace{}%
\AgdaField{walks}\AgdaSpace{}%
\AgdaSymbol{=}\AgdaSpace{}%
\AgdaFunction{helper}\AgdaSpace{}%
\AgdaSymbol{\}}\<%
\\
\>[2][@{}l@{\AgdaIndent{0}}]%
\>[4]\AgdaKeyword{where}\<%
\\
\>[4][@{}l@{\AgdaIndent{0}}]%
\>[6]\AgdaFunction{helper}\AgdaSpace{}%
\AgdaSymbol{:}\AgdaSpace{}%
\AgdaRecord{irishThing}\AgdaSpace{}%
\AgdaPostulate{animal}\AgdaSpace{}%
\AgdaSymbol{→}\AgdaSpace{}%
\AgdaPrimitive{Set}\<%
\\
%
\>[6]\AgdaFunction{helper}\AgdaSpace{}%
\AgdaSymbol{(}\AgdaInductiveConstructor{mkIrish}\AgdaSpace{}%
\AgdaBound{a}\AgdaSpace{}%
\AgdaBound{isIrish₁}\AgdaSymbol{)}\AgdaSpace{}%
\AgdaSymbol{=}\AgdaSpace{}%
\AgdaField{Walks.walks}\AgdaSpace{}%
\AgdaPostulate{animalsWalk}\AgdaSpace{}%
\AgdaBound{a}\<%
\\
%
\\[\AgdaEmptyExtraSkip]%
%
\>[2]\AgdaFunction{irishHumanWalks}\AgdaSpace{}%
\AgdaSymbol{:}\AgdaSpace{}%
\AgdaRecord{Walks}\AgdaSpace{}%
\AgdaFunction{IrishHuman}\<%
\\
%
\>[2]\AgdaFunction{irishHumanWalks}\AgdaSpace{}%
\AgdaSymbol{=}\AgdaSpace{}%
\AgdaKeyword{record}\AgdaSpace{}%
\AgdaSymbol{\{}\AgdaSpace{}%
\AgdaField{walks}\AgdaSpace{}%
\AgdaSymbol{=}\AgdaSpace{}%
\AgdaFunction{helper}\AgdaSpace{}%
\AgdaSymbol{\}}\<%
\\
\>[2][@{}l@{\AgdaIndent{0}}]%
\>[4]\AgdaKeyword{where}\<%
\\
\>[4][@{}l@{\AgdaIndent{0}}]%
\>[6]\AgdaFunction{helper}\AgdaSpace{}%
\AgdaSymbol{:}\AgdaSpace{}%
\AgdaRecord{irishThing}\AgdaSpace{}%
\AgdaPostulate{human}\AgdaSpace{}%
\AgdaSymbol{→}\AgdaSpace{}%
\AgdaPrimitive{Set}\<%
\\
%
\>[6]\AgdaFunction{helper}\AgdaSpace{}%
\AgdaSymbol{(}\AgdaInductiveConstructor{mkIrish}\AgdaSpace{}%
\AgdaBound{a}\AgdaSpace{}%
\AgdaBound{isIrish₁}\AgdaSymbol{)}\AgdaSpace{}%
\AgdaSymbol{=}\AgdaSpace{}%
\AgdaField{Walks.walks}\AgdaSpace{}%
\AgdaPostulate{animalsWalk}\AgdaSpace{}%
\AgdaOperator{\AgdaFunction{\$}}\AgdaSpace{}%
\AgdaBound{a}\<%
\end{code}
\begin{code}[hide]%
%
\>[2]\AgdaFunction{irishDelegateWalks}\AgdaSpace{}%
\AgdaSymbol{:}\AgdaSpace{}%
\AgdaRecord{Walks}\AgdaSpace{}%
\AgdaFunction{IrishDelegate}\<%
\\
%
\>[2]\AgdaFunction{irishDelegateWalks}\AgdaSpace{}%
\AgdaSymbol{=}\AgdaSpace{}%
\AgdaKeyword{record}\AgdaSpace{}%
\AgdaSymbol{\{}\AgdaSpace{}%
\AgdaField{walks}\AgdaSpace{}%
\AgdaSymbol{=}\AgdaSpace{}%
\AgdaFunction{helper}\AgdaSpace{}%
\AgdaSymbol{\}}\<%
\\
\>[2][@{}l@{\AgdaIndent{0}}]%
\>[4]\AgdaKeyword{where}\<%
\\
\>[4][@{}l@{\AgdaIndent{0}}]%
\>[6]\AgdaFunction{helper}\AgdaSpace{}%
\AgdaSymbol{:}\AgdaSpace{}%
\AgdaRecord{irishThing}\AgdaSpace{}%
\AgdaPostulate{delegate}\AgdaSpace{}%
\AgdaSymbol{→}\AgdaSpace{}%
\AgdaPrimitive{Set}\<%
\\
%
\>[6]\AgdaFunction{helper}\AgdaSpace{}%
\AgdaSymbol{(}\AgdaInductiveConstructor{mkIrish}\AgdaSpace{}%
\AgdaBound{d}\AgdaSpace{}%
\AgdaBound{isIrish₁}\AgdaSymbol{)}\AgdaSpace{}%
\AgdaSymbol{=}\AgdaSpace{}%
\AgdaField{Walks.walks}\AgdaSpace{}%
\AgdaPostulate{animalsWalk}\AgdaSpace{}%
\AgdaOperator{\AgdaFunction{\$}}\AgdaSpace{}%
\AgdaBound{d}\<%
\\
%
\\[\AgdaEmptyExtraSkip]%
\>[0]\AgdaFunction{thm?}\AgdaSpace{}%
\AgdaSymbol{:}\AgdaSpace{}%
\AgdaFunction{some}\AgdaSpace{}%
\AgdaFunction{IrishDelegate}\AgdaSpace{}%
\AgdaField{walks}%
\>[33]\AgdaSymbol{→}\AgdaSpace{}%
\AgdaFunction{some}\AgdaSpace{}%
\AgdaFunction{IrishHuman}\AgdaSpace{}%
\AgdaField{walks}\<%
\\
\>[0]\AgdaFunction{thm?}\AgdaSpace{}%
\AgdaSymbol{(}\AgdaInductiveConstructor{mkIrish}\AgdaSpace{}%
\AgdaBound{del}\AgdaSpace{}%
\AgdaBound{isIrish[del]}\AgdaSpace{}%
\AgdaOperator{\AgdaInductiveConstructor{,}}\AgdaSpace{}%
\AgdaBound{snd}\AgdaSymbol{)}\AgdaSpace{}%
\AgdaSymbol{=}\AgdaSpace{}%
\AgdaSymbol{(}\AgdaInductiveConstructor{mkIrish}\AgdaSpace{}%
\AgdaSymbol{(}\AgdaPostulate{dh}\AgdaSpace{}%
\AgdaBound{del}\AgdaSymbol{)}\AgdaSpace{}%
\AgdaBound{isIrish[del]}\AgdaSymbol{)}\AgdaSpace{}%
\AgdaOperator{\AgdaInductiveConstructor{,}}\AgdaSpace{}%
\AgdaBound{snd}\<%
\\
%
\\[\AgdaEmptyExtraSkip]%
\>[0]\AgdaFunction{id}\AgdaSpace{}%
\AgdaSymbol{:}\AgdaSpace{}%
\AgdaSymbol{\{}\AgdaBound{A}\AgdaSpace{}%
\AgdaSymbol{:}\AgdaSpace{}%
\AgdaPrimitive{Set}\AgdaSymbol{\}}\AgdaSpace{}%
\AgdaSymbol{→}\AgdaSpace{}%
\AgdaBound{A}\AgdaSpace{}%
\AgdaSymbol{→}\AgdaSpace{}%
\AgdaBound{A}\<%
\\
\>[0]\AgdaFunction{id}\AgdaSpace{}%
\AgdaBound{x}\AgdaSpace{}%
\AgdaSymbol{=}\AgdaSpace{}%
\AgdaBound{x}\<%
\end{code}

We can now prove analagous theorems to what we showed earlier, with the
adjectival modification showing as extra data in both the input and output. One
can always forsake the Irish detail and prove a weaker conclusion, as in
\term{thm5}.

\begin{code}%
\>[0]\AgdaFunction{thm4}\AgdaSpace{}%
\AgdaSymbol{:}\AgdaSpace{}%
\AgdaFunction{some}\AgdaSpace{}%
\AgdaFunction{IrishHuman}\AgdaSpace{}%
\AgdaField{walks}\AgdaSpace{}%
\AgdaSymbol{→}\AgdaSpace{}%
\AgdaFunction{some}\AgdaSpace{}%
\AgdaFunction{IrishAnimal}\AgdaSpace{}%
\AgdaField{walks}\<%
\\
\>[0]\AgdaFunction{thm4}\AgdaSpace{}%
\AgdaSymbol{(}\AgdaInductiveConstructor{mkIrish}\AgdaSpace{}%
\AgdaBound{hum}\AgdaSpace{}%
\AgdaBound{isIrish[hum]}\AgdaSpace{}%
\AgdaOperator{\AgdaInductiveConstructor{,}}\AgdaSpace{}%
\AgdaBound{snd}\AgdaSymbol{)}\AgdaSpace{}%
\AgdaSymbol{=}\AgdaSpace{}%
\AgdaSymbol{(}\AgdaInductiveConstructor{mkIrish}\AgdaSpace{}%
\AgdaSymbol{(}\AgdaPostulate{ha}\AgdaSpace{}%
\AgdaBound{hum}\AgdaSymbol{)}\AgdaSpace{}%
\AgdaBound{isIrish[hum]}\AgdaSymbol{)}\AgdaSpace{}%
\AgdaOperator{\AgdaInductiveConstructor{,}}\AgdaSpace{}%
\AgdaBound{snd}\<%
\\
%
\\[\AgdaEmptyExtraSkip]%
\>[0]\AgdaFunction{thm5}\AgdaSpace{}%
\AgdaSymbol{:}\AgdaSpace{}%
\AgdaFunction{some}\AgdaSpace{}%
\AgdaFunction{IrishHuman}\AgdaSpace{}%
\AgdaField{walks}\AgdaSpace{}%
\AgdaSymbol{→}\AgdaSpace{}%
\AgdaFunction{some}\AgdaSpace{}%
\AgdaPostulate{animal}\AgdaSpace{}%
\AgdaField{walks}\<%
\\
\>[0]\AgdaFunction{thm5}\AgdaSpace{}%
\AgdaSymbol{(}\AgdaInductiveConstructor{mkIrish}\AgdaSpace{}%
\AgdaBound{hum}\AgdaSpace{}%
\AgdaBound{isIrish[hum]}\AgdaSpace{}%
\AgdaOperator{\AgdaInductiveConstructor{,}}\AgdaSpace{}%
\AgdaBound{snd}\AgdaSymbol{)}\AgdaSpace{}%
\AgdaSymbol{=}\AgdaSpace{}%
\AgdaSymbol{(}\AgdaPostulate{ha}\AgdaSpace{}%
\AgdaBound{hum}\AgdaSymbol{)}\AgdaSpace{}%
\AgdaOperator{\AgdaInductiveConstructor{,}}\AgdaSpace{}%
\AgdaBound{snd}\<%
\end{code}

If we now decide to now assume some anonymous \term{irishHuman} exists, and we
prove that human is an animal in \term{irishAnimal}, we can see the fruits of
our labor insofar as the identity funtion works in \term{thm6} despite the
property of our specimin walking being of different types. In \term{thm7}, we
can also then use our anonymous human as a witness for the existential claim
that ``some Irish animal walks".

\begin{code}%
\>[0]\AgdaKeyword{postulate}\<%
\\
\>[0][@{}l@{\AgdaIndent{0}}]%
\>[2]\AgdaPostulate{irishHuman}\AgdaSpace{}%
\AgdaSymbol{:}\AgdaSpace{}%
\AgdaRecord{irishThing}\AgdaSpace{}%
\AgdaPostulate{human}\AgdaSpace{}%
\AgdaSymbol{\{\{}\AgdaFunction{hoc}\AgdaSymbol{\}\}}\<%
\\
%
\\[\AgdaEmptyExtraSkip]%
\>[0]\AgdaKeyword{instance}\<%
\\
\>[0][@{}l@{\AgdaIndent{0}}]%
\>[2]\AgdaFunction{irishAnimal}\AgdaSpace{}%
\AgdaSymbol{:}\AgdaSpace{}%
\AgdaRecord{irishThing}\AgdaSpace{}%
\AgdaPostulate{animal}\<%
\\
%
\>[2]\AgdaFunction{irishAnimal}\AgdaSpace{}%
\AgdaSymbol{=}\AgdaSpace{}%
\AgdaInductiveConstructor{mkIrish}\AgdaSpace{}%
\AgdaSymbol{(}\AgdaPostulate{ha}\AgdaSpace{}%
\AgdaSymbol{(}\AgdaField{irishThing.thing}\AgdaSpace{}%
\AgdaPostulate{irishHuman}\AgdaSymbol{))}\AgdaSpace{}%
\AgdaSymbol{(}\AgdaField{irishThing.isIrish}\AgdaSpace{}%
\AgdaPostulate{irishHuman}\AgdaSymbol{)}\<%
\\
%
\\[\AgdaEmptyExtraSkip]%
\>[0]\AgdaFunction{thm6}\AgdaSpace{}%
\AgdaSymbol{:}\AgdaSpace{}%
\AgdaField{walks}\AgdaSpace{}%
\AgdaFunction{irishAnimal}\AgdaSpace{}%
\AgdaSymbol{→}\AgdaSpace{}%
\AgdaField{walks}\AgdaSpace{}%
\AgdaPostulate{irishHuman}\<%
\\
\>[0]\AgdaFunction{thm6}\AgdaSpace{}%
\AgdaSymbol{=}\AgdaSpace{}%
\AgdaFunction{id}\<%
\\
%
\\[\AgdaEmptyExtraSkip]%
\>[0]\AgdaFunction{thm7}\AgdaSpace{}%
\AgdaSymbol{:}\AgdaSpace{}%
\AgdaField{walks}\AgdaSpace{}%
\AgdaPostulate{irishHuman}\AgdaSpace{}%
\AgdaSymbol{→}\AgdaSpace{}%
\AgdaFunction{some}\AgdaSpace{}%
\AgdaFunction{IrishAnimal}\AgdaSpace{}%
\AgdaField{walks}\<%
\\
\>[0]\AgdaFunction{thm7}\AgdaSpace{}%
\AgdaBound{x}\AgdaSpace{}%
\AgdaSymbol{=}\<%
\\
\>[0][@{}l@{\AgdaIndent{0}}]%
\>[2]\AgdaInductiveConstructor{mkIrish}\AgdaSpace{}%
\AgdaSymbol{(}\AgdaPostulate{ha}\AgdaSpace{}%
\AgdaSymbol{(}\AgdaField{irishThing.thing}\AgdaSpace{}%
\AgdaPostulate{irishHuman}\AgdaSymbol{))}\AgdaSpace{}%
\AgdaSymbol{((}\AgdaField{irishThing.isIrish}\AgdaSpace{}%
\AgdaPostulate{irishHuman}\AgdaSymbol{))}\AgdaSpace{}%
\AgdaOperator{\AgdaInductiveConstructor{,}}\AgdaSpace{}%
\AgdaBound{x}\<%
\end{code}

One might try to prove something even sillier, like that an Irish animal is an
Irish thing object. Problematically, for the instance checker to be happy, we need to reflexivly coerce an object due to the constraint that a coercion to an object must exist to build and \term{irishThing}. This then makes it so that if we want to 

\begin{code}%
\>[0]\AgdaKeyword{postulate}\<%
\\
\>[0][@{}l@{\AgdaIndent{0}}]%
\>[2]\AgdaPostulate{oo}\AgdaSpace{}%
\AgdaSymbol{:}\AgdaSpace{}%
\AgdaPostulate{object}\AgdaSpace{}%
\AgdaSymbol{→}\AgdaSpace{}%
\AgdaPostulate{object}\<%
\\
\>[0]\AgdaKeyword{instance}\<%
\\
\>[0][@{}l@{\AgdaIndent{0}}]%
\>[2]\AgdaFunction{ooc}\AgdaSpace{}%
\AgdaSymbol{=}\AgdaSpace{}%
\AgdaOperator{\AgdaInductiveConstructor{⌞}}\AgdaSpace{}%
\AgdaPostulate{oo}\AgdaSpace{}%
\AgdaOperator{\AgdaInductiveConstructor{⌟}}\<%
\\
%
\\[\AgdaEmptyExtraSkip]%
\>[0]\AgdaKeyword{postulate}\<%
\\
\>[0][@{}l@{\AgdaIndent{0}}]%
\>[2]\AgdaPostulate{irishHumanisIrishThing}\AgdaSpace{}%
\AgdaSymbol{:}\AgdaSpace{}%
\AgdaRecord{irishThing}\AgdaSpace{}%
\AgdaPostulate{animal}\AgdaSpace{}%
\AgdaSymbol{→}\AgdaSpace{}%
\AgdaRecord{irishThing}\AgdaSpace{}%
\AgdaPostulate{object}\<%
\end{code}

If one tries to prove this though, it's impossible to complete the program.

\begin{verbatim}
irishHumanisIrishThing (mkIrish thing isIrish) = mkIrish ((ao (thing))) {!!}
\end{verbatim}

Agda computes with the reflexive coercion instance, and therefore we come to the unredeemable goal :

\begin{verbatim}
Goal: irish $ ao thing
Have: irish $ thing
\end{verbatim}

One might think to just add an extra instance to appease Agda :

\begin{code}%
\>[0]\AgdaKeyword{instance}\<%
\\
\>[0][@{}l@{\AgdaIndent{0}}]%
\>[2]\AgdaFunction{aooc}\AgdaSpace{}%
\AgdaSymbol{=}\AgdaSpace{}%
\AgdaOperator{\AgdaInductiveConstructor{⌞}}\AgdaSpace{}%
\AgdaPostulate{ao}\AgdaSpace{}%
\AgdaOperator{\AgdaInductiveConstructor{⌟}}\AgdaSpace{}%
\AgdaOperator{\AgdaFunction{⊚}}\AgdaSpace{}%
\AgdaFunction{ooc}\<%
\end{code}

However, if we were to add an additional instance allowing an animal to be
coerced to an object, this would break the necessary uniqueness of instance
arguements, consistent with the uniqueness of coercions property in type
theories supporting coercive subtyping. This example highlights the limitations
of working with a make-believe subtyping mechanism. While instances give the
Agda programmer the benefits of ad-hoc polymorphism, they are is still not a substitute for a type theory with coercive subtyping built in, especially when it comes to MTT semantics.

\subsection{Addendum on Inductive Types}

\begin{code}%
\>[0]\AgdaKeyword{data}\AgdaSpace{}%
\AgdaDatatype{Men}\AgdaSpace{}%
\AgdaSymbol{:}\AgdaSpace{}%
\AgdaFunction{CN}\AgdaSpace{}%
\AgdaKeyword{where}\<%
\\
\>[0][@{}l@{\AgdaIndent{0}}]%
\>[2]\AgdaInductiveConstructor{Steve}\AgdaSpace{}%
\AgdaSymbol{:}\AgdaSpace{}%
\AgdaDatatype{Men}\<%
\\
%
\>[2]\AgdaInductiveConstructor{Dave}\AgdaSpace{}%
\AgdaSymbol{:}\AgdaSpace{}%
\AgdaDatatype{Men}\<%
\\
%
\\[\AgdaEmptyExtraSkip]%
\>[0]\AgdaKeyword{data}\AgdaSpace{}%
\AgdaDatatype{Human}\AgdaSpace{}%
\AgdaSymbol{:}\AgdaSpace{}%
\AgdaFunction{CN}\AgdaSpace{}%
\AgdaKeyword{where}\<%
\\
\>[0][@{}l@{\AgdaIndent{0}}]%
\>[2]\AgdaInductiveConstructor{MenHuman}\AgdaSpace{}%
\AgdaSymbol{:}\AgdaSpace{}%
\AgdaDatatype{Men}\AgdaSpace{}%
\AgdaSymbol{→}\AgdaSpace{}%
\AgdaDatatype{Human}\<%
\\
%
\\[\AgdaEmptyExtraSkip]%
\>[0]\AgdaFunction{SteveHuman}\AgdaSpace{}%
\AgdaSymbol{:}\AgdaSpace{}%
\AgdaDatatype{Human}\<%
\\
\>[0]\AgdaFunction{SteveHuman}\AgdaSpace{}%
\AgdaSymbol{=}\AgdaSpace{}%
\AgdaInductiveConstructor{MenHuman}\AgdaSpace{}%
\AgdaInductiveConstructor{Steve}\<%
\\
%
\\[\AgdaEmptyExtraSkip]%
\>[0]\AgdaComment{-- what if we map this to actual evidence}\<%
\\
\>[0]\AgdaFunction{Walk}\AgdaSpace{}%
\AgdaSymbol{:}\AgdaSpace{}%
\AgdaDatatype{Human}\AgdaSpace{}%
\AgdaSymbol{→}\AgdaSpace{}%
\AgdaPrimitive{Set}\<%
\\
\>[0]\AgdaFunction{Walk}\AgdaSpace{}%
\AgdaSymbol{(}\AgdaInductiveConstructor{MenHuman}\AgdaSpace{}%
\AgdaInductiveConstructor{Steve}\AgdaSymbol{)}\AgdaSpace{}%
\AgdaSymbol{=}\AgdaSpace{}%
\AgdaRecord{⊤}\<%
\\
\>[0]\AgdaFunction{Walk}\AgdaSpace{}%
\AgdaSymbol{(}\AgdaInductiveConstructor{MenHuman}\AgdaSpace{}%
\AgdaInductiveConstructor{Dave}\AgdaSymbol{)}\AgdaSpace{}%
\AgdaSymbol{=}\AgdaSpace{}%
\AgdaRecord{⊤}\<%
\\
%
\\[\AgdaEmptyExtraSkip]%
\>[0]\AgdaFunction{allmenWalk}\AgdaSpace{}%
\AgdaSymbol{:}\AgdaSpace{}%
\AgdaFunction{all}\AgdaSpace{}%
\AgdaDatatype{Men}\AgdaSpace{}%
\AgdaSymbol{λ}\AgdaSpace{}%
\AgdaBound{x}\AgdaSpace{}%
\AgdaSymbol{→}\AgdaSpace{}%
\AgdaFunction{Walk}\AgdaSpace{}%
\AgdaSymbol{(}\AgdaInductiveConstructor{MenHuman}\AgdaSpace{}%
\AgdaBound{x}\AgdaSymbol{)}\<%
\\
\>[0]\AgdaFunction{allmenWalk}\AgdaSpace{}%
\AgdaInductiveConstructor{Steve}\AgdaSpace{}%
\AgdaSymbol{=}\AgdaSpace{}%
\AgdaInductiveConstructor{tt}\<%
\\
\>[0]\AgdaFunction{allmenWalk}\AgdaSpace{}%
\AgdaInductiveConstructor{Dave}\AgdaSpace{}%
\AgdaSymbol{=}\AgdaSpace{}%
\AgdaInductiveConstructor{tt}\<%
\\
%
\\[\AgdaEmptyExtraSkip]%
\>[0]\AgdaKeyword{data}\AgdaSpace{}%
\AgdaDatatype{⊥}\AgdaSpace{}%
\AgdaSymbol{:}\AgdaSpace{}%
\AgdaPrimitive{Set}\AgdaSpace{}%
\AgdaKeyword{where}\<%
\\
%
\\[\AgdaEmptyExtraSkip]%
\>[0]\AgdaFunction{Walk'}\AgdaSpace{}%
\AgdaSymbol{:}\AgdaSpace{}%
\AgdaDatatype{Human}\AgdaSpace{}%
\AgdaSymbol{→}\AgdaSpace{}%
\AgdaPrimitive{Set}\<%
\\
\>[0]\AgdaFunction{Walk'}\AgdaSpace{}%
\AgdaSymbol{(}\AgdaInductiveConstructor{MenHuman}\AgdaSpace{}%
\AgdaInductiveConstructor{Steve}\AgdaSymbol{)}\AgdaSpace{}%
\AgdaSymbol{=}\AgdaSpace{}%
\AgdaRecord{⊤}\<%
\\
\>[0]\AgdaFunction{Walk'}\AgdaSpace{}%
\AgdaSymbol{(}\AgdaInductiveConstructor{MenHuman}\AgdaSpace{}%
\AgdaInductiveConstructor{Dave}\AgdaSymbol{)}\AgdaSpace{}%
\AgdaSymbol{=}\AgdaSpace{}%
\AgdaDatatype{⊥}\<%
\\
%
\\[\AgdaEmptyExtraSkip]%
\>[0]\AgdaFunction{someManWalks'}\AgdaSpace{}%
\AgdaSymbol{:}\AgdaSpace{}%
\AgdaFunction{some}\AgdaSpace{}%
\AgdaDatatype{Men}\AgdaSpace{}%
\AgdaSymbol{λ}\AgdaSpace{}%
\AgdaBound{x}\AgdaSpace{}%
\AgdaSymbol{→}\AgdaSpace{}%
\AgdaFunction{Walk'}\AgdaSpace{}%
\AgdaSymbol{(}\AgdaInductiveConstructor{MenHuman}\AgdaSpace{}%
\AgdaBound{x}\AgdaSymbol{)}\<%
\\
\>[0]\AgdaField{proj₁}\AgdaSpace{}%
\AgdaFunction{someManWalks'}\AgdaSpace{}%
\AgdaSymbol{=}\AgdaSpace{}%
\AgdaInductiveConstructor{Steve}\<%
\\
\>[0]\AgdaField{proj₂}\AgdaSpace{}%
\AgdaFunction{someManWalks'}\AgdaSpace{}%
\AgdaSymbol{=}\AgdaSpace{}%
\AgdaInductiveConstructor{tt}\<%
\\
%
\\[\AgdaEmptyExtraSkip]%
\>[0]\AgdaComment{-- allmenDontWalk : all Men λ x → Walk' (MenHuman x)}\<%
\\
\>[0]\AgdaComment{-- allmenDontWalk Steve = tt}\<%
\\
\>[0]\AgdaComment{-- allmenDontWalk Dave = \{!!\}}\<%
\\
%
\\[\AgdaEmptyExtraSkip]%
\>[0]\AgdaFunction{steveWalks}\AgdaSpace{}%
\AgdaSymbol{:}\AgdaSpace{}%
\AgdaFunction{Walk}\AgdaSpace{}%
\AgdaSymbol{(}\AgdaInductiveConstructor{MenHuman}\AgdaSpace{}%
\AgdaInductiveConstructor{Steve}\AgdaSymbol{)}\AgdaSpace{}%
\AgdaComment{-- : all Men λ x → Walk (MenHuman x)}\<%
\\
\>[0]\AgdaFunction{steveWalks}\AgdaSpace{}%
\AgdaSymbol{=}\AgdaSpace{}%
\AgdaInductiveConstructor{tt}\<%
\\
%
\\[\AgdaEmptyExtraSkip]%
\>[0]\AgdaFunction{someManWalks}\AgdaSpace{}%
\AgdaSymbol{:}\AgdaSpace{}%
\AgdaFunction{some}\AgdaSpace{}%
\AgdaDatatype{Men}\AgdaSpace{}%
\AgdaSymbol{λ}\AgdaSpace{}%
\AgdaBound{x}\AgdaSpace{}%
\AgdaSymbol{→}\AgdaSpace{}%
\AgdaFunction{Walk}\AgdaSpace{}%
\AgdaSymbol{(}\AgdaInductiveConstructor{MenHuman}\AgdaSpace{}%
\AgdaBound{x}\AgdaSymbol{)}\<%
\\
\>[0]\AgdaField{proj₁}\AgdaSpace{}%
\AgdaFunction{someManWalks}\AgdaSpace{}%
\AgdaSymbol{=}\AgdaSpace{}%
\AgdaInductiveConstructor{Steve}\<%
\\
\>[0]\AgdaField{proj₂}\AgdaSpace{}%
\AgdaFunction{someManWalks}\AgdaSpace{}%
\AgdaSymbol{=}\AgdaSpace{}%
\AgdaInductiveConstructor{tt}\<%
\end{code}


\section{Conclusion and Conclusion}

\printbibliography

\end{document}
